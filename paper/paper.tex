% Options for packages loaded elsewhere
\PassOptionsToPackage{unicode}{hyperref}
\PassOptionsToPackage{hyphens}{url}
%
\documentclass[
]{article}
\usepackage{lmodern}
\usepackage{amsmath}
\usepackage{ifxetex,ifluatex}
\ifnum 0\ifxetex 1\fi\ifluatex 1\fi=0 % if pdftex
  \usepackage[T1]{fontenc}
  \usepackage[utf8]{inputenc}
  \usepackage{textcomp} % provide euro and other symbols
  \usepackage{amssymb}
\else % if luatex or xetex
  \usepackage{unicode-math}
  \defaultfontfeatures{Scale=MatchLowercase}
  \defaultfontfeatures[\rmfamily]{Ligatures=TeX,Scale=1}
\fi
% Use upquote if available, for straight quotes in verbatim environments
\IfFileExists{upquote.sty}{\usepackage{upquote}}{}
\IfFileExists{microtype.sty}{% use microtype if available
  \usepackage[]{microtype}
  \UseMicrotypeSet[protrusion]{basicmath} % disable protrusion for tt fonts
}{}
\makeatletter
\@ifundefined{KOMAClassName}{% if non-KOMA class
  \IfFileExists{parskip.sty}{%
    \usepackage{parskip}
  }{% else
    \setlength{\parindent}{0pt}
    \setlength{\parskip}{6pt plus 2pt minus 1pt}}
}{% if KOMA class
  \KOMAoptions{parskip=half}}
\makeatother
\usepackage{xcolor}
\IfFileExists{xurl.sty}{\usepackage{xurl}}{} % add URL line breaks if available
\IfFileExists{bookmark.sty}{\usepackage{bookmark}}{\usepackage{hyperref}}
\hypersetup{
  pdftitle={Nest Provisioning in a Fire Disturbed Landscape},
  pdfauthor={Eliza Stein},
  hidelinks,
  pdfcreator={LaTeX via pandoc}}
\urlstyle{same} % disable monospaced font for URLs
\usepackage[margin=1in]{geometry}
\usepackage{graphicx}
\makeatletter
\def\maxwidth{\ifdim\Gin@nat@width>\linewidth\linewidth\else\Gin@nat@width\fi}
\def\maxheight{\ifdim\Gin@nat@height>\textheight\textheight\else\Gin@nat@height\fi}
\makeatother
% Scale images if necessary, so that they will not overflow the page
% margins by default, and it is still possible to overwrite the defaults
% using explicit options in \includegraphics[width, height, ...]{}
\setkeys{Gin}{width=\maxwidth,height=\maxheight,keepaspectratio}
% Set default figure placement to htbp
\makeatletter
\def\fps@figure{htbp}
\makeatother
\setlength{\emergencystretch}{3em} % prevent overfull lines
\providecommand{\tightlist}{%
  \setlength{\itemsep}{0pt}\setlength{\parskip}{0pt}}
\setcounter{secnumdepth}{-\maxdimen} % remove section numbering
\ifluatex
  \usepackage{selnolig}  % disable illegal ligatures
\fi
\usepackage[]{natbib}
\bibliographystyle{plainnat}

\title{Nest Provisioning in a Fire Disturbed Landscape}
\author{Eliza Stein}
\date{11/8/2020}

\begin{document}
\maketitle

\hypertarget{introduction}{%
\section{Introduction}\label{introduction}}

Fire plays an important role as a consistent disturbance in maintaining
open stands of old-growth Ponderosa Pine (\emph{Pinus ponderosa})
forests by helping to eliminate understory and limit fuel loads
\citep{veblen2000climatic}. Before human intervention, Ponderosa Pine
forests naturally underwent forest fires in 5-50 year intervals
\citep{veblen2000climatic}. Over the past century, however, tree
planting initiatives and increased implementation of fire suppression
have led to increased density of stands \citep{griffis2001understory},
making forest stands that are already drought stressed even more
susceptible to high severity crown fires \citep{veblen2000climatic}. In
2002, a human-caused wildfire, the Hayman Fire, burned 138,000 acres of
old-growth Ponderosa pine forests in Colorado's Pike National Forest
\citep{graham2003hayman}.

The Flammulated Owl (\emph{Psiloscops flammeolus}) is a territorial,
insectivorous, and nocturnal raptor native to montane forests in
portions of the Rocky Mountains, Sierra Nevada Mountains, and the
Occidental Mountains \citep{linkhart2013flammulated}. The diet of the
owl primarily consists of moths native to these regions
\citep{linkhart2013flammulated}. As a highly specialized secondary
cavity nesting raptor, the Flammulated Owl is deemed an indicator
species, meaning that the health of an ecosystem can be estimated based
on the health of their population. Survival models have shown that
Flammulated Owl survival in the HFSA is currently lower than survival in
MGSA, suggesting that mortality, rather than emigration, explains most
of the population declines following the Hayman Fire (Linkhart and
Yanco, unpublished data).

Here, I examine one possible explanation for increased mortality in
HFSA: prey availability. High severity burns dramatically alter
vegetation structure, which in turn alters insect communities. Over
time, insect communities within high intensity burn scars can crash,
leaving avian predators without important food resources
\citep{nappi2010effect}. If Flammulated Owls are adapting their behavior
in response to changing prey availability, I would expect that the rate
of prey deliveries to active nests would increase or decrease (increase
if prey is lower quality, decrease if prey is more scarce or difficult
to detect) \citep{zarybnicka2009tengmalm}. If Flammulated Owls are not
adapting their behavior, this could mean that prey availability has
either not changed or, more likely, that Flammulated Owls, which do not
occupy landscapes prone to high severity burns, do not adapt their
behavior in response to large-scale landscape changes.

  \bibliography{references.bib}

\end{document}
