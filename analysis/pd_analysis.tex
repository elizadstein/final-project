% Options for packages loaded elsewhere
\PassOptionsToPackage{unicode}{hyperref}
\PassOptionsToPackage{hyphens}{url}
%
\documentclass[
]{article}
\usepackage{lmodern}
\usepackage{amsmath}
\usepackage{ifxetex,ifluatex}
\ifnum 0\ifxetex 1\fi\ifluatex 1\fi=0 % if pdftex
  \usepackage[T1]{fontenc}
  \usepackage[utf8]{inputenc}
  \usepackage{textcomp} % provide euro and other symbols
  \usepackage{amssymb}
\else % if luatex or xetex
  \usepackage{unicode-math}
  \defaultfontfeatures{Scale=MatchLowercase}
  \defaultfontfeatures[\rmfamily]{Ligatures=TeX,Scale=1}
\fi
% Use upquote if available, for straight quotes in verbatim environments
\IfFileExists{upquote.sty}{\usepackage{upquote}}{}
\IfFileExists{microtype.sty}{% use microtype if available
  \usepackage[]{microtype}
  \UseMicrotypeSet[protrusion]{basicmath} % disable protrusion for tt fonts
}{}
\makeatletter
\@ifundefined{KOMAClassName}{% if non-KOMA class
  \IfFileExists{parskip.sty}{%
    \usepackage{parskip}
  }{% else
    \setlength{\parindent}{0pt}
    \setlength{\parskip}{6pt plus 2pt minus 1pt}}
}{% if KOMA class
  \KOMAoptions{parskip=half}}
\makeatother
\usepackage{xcolor}
\IfFileExists{xurl.sty}{\usepackage{xurl}}{} % add URL line breaks if available
\IfFileExists{bookmark.sty}{\usepackage{bookmark}}{\usepackage{hyperref}}
\hypersetup{
  pdftitle={Nest Provisioning in a Fire Disturbed Landscape},
  pdfauthor={Eliza Stein},
  hidelinks,
  pdfcreator={LaTeX via pandoc}}
\urlstyle{same} % disable monospaced font for URLs
\usepackage[margin=1in]{geometry}
\usepackage{color}
\usepackage{fancyvrb}
\newcommand{\VerbBar}{|}
\newcommand{\VERB}{\Verb[commandchars=\\\{\}]}
\DefineVerbatimEnvironment{Highlighting}{Verbatim}{commandchars=\\\{\}}
% Add ',fontsize=\small' for more characters per line
\usepackage{framed}
\definecolor{shadecolor}{RGB}{248,248,248}
\newenvironment{Shaded}{\begin{snugshade}}{\end{snugshade}}
\newcommand{\AlertTok}[1]{\textcolor[rgb]{0.94,0.16,0.16}{#1}}
\newcommand{\AnnotationTok}[1]{\textcolor[rgb]{0.56,0.35,0.01}{\textbf{\textit{#1}}}}
\newcommand{\AttributeTok}[1]{\textcolor[rgb]{0.77,0.63,0.00}{#1}}
\newcommand{\BaseNTok}[1]{\textcolor[rgb]{0.00,0.00,0.81}{#1}}
\newcommand{\BuiltInTok}[1]{#1}
\newcommand{\CharTok}[1]{\textcolor[rgb]{0.31,0.60,0.02}{#1}}
\newcommand{\CommentTok}[1]{\textcolor[rgb]{0.56,0.35,0.01}{\textit{#1}}}
\newcommand{\CommentVarTok}[1]{\textcolor[rgb]{0.56,0.35,0.01}{\textbf{\textit{#1}}}}
\newcommand{\ConstantTok}[1]{\textcolor[rgb]{0.00,0.00,0.00}{#1}}
\newcommand{\ControlFlowTok}[1]{\textcolor[rgb]{0.13,0.29,0.53}{\textbf{#1}}}
\newcommand{\DataTypeTok}[1]{\textcolor[rgb]{0.13,0.29,0.53}{#1}}
\newcommand{\DecValTok}[1]{\textcolor[rgb]{0.00,0.00,0.81}{#1}}
\newcommand{\DocumentationTok}[1]{\textcolor[rgb]{0.56,0.35,0.01}{\textbf{\textit{#1}}}}
\newcommand{\ErrorTok}[1]{\textcolor[rgb]{0.64,0.00,0.00}{\textbf{#1}}}
\newcommand{\ExtensionTok}[1]{#1}
\newcommand{\FloatTok}[1]{\textcolor[rgb]{0.00,0.00,0.81}{#1}}
\newcommand{\FunctionTok}[1]{\textcolor[rgb]{0.00,0.00,0.00}{#1}}
\newcommand{\ImportTok}[1]{#1}
\newcommand{\InformationTok}[1]{\textcolor[rgb]{0.56,0.35,0.01}{\textbf{\textit{#1}}}}
\newcommand{\KeywordTok}[1]{\textcolor[rgb]{0.13,0.29,0.53}{\textbf{#1}}}
\newcommand{\NormalTok}[1]{#1}
\newcommand{\OperatorTok}[1]{\textcolor[rgb]{0.81,0.36,0.00}{\textbf{#1}}}
\newcommand{\OtherTok}[1]{\textcolor[rgb]{0.56,0.35,0.01}{#1}}
\newcommand{\PreprocessorTok}[1]{\textcolor[rgb]{0.56,0.35,0.01}{\textit{#1}}}
\newcommand{\RegionMarkerTok}[1]{#1}
\newcommand{\SpecialCharTok}[1]{\textcolor[rgb]{0.00,0.00,0.00}{#1}}
\newcommand{\SpecialStringTok}[1]{\textcolor[rgb]{0.31,0.60,0.02}{#1}}
\newcommand{\StringTok}[1]{\textcolor[rgb]{0.31,0.60,0.02}{#1}}
\newcommand{\VariableTok}[1]{\textcolor[rgb]{0.00,0.00,0.00}{#1}}
\newcommand{\VerbatimStringTok}[1]{\textcolor[rgb]{0.31,0.60,0.02}{#1}}
\newcommand{\WarningTok}[1]{\textcolor[rgb]{0.56,0.35,0.01}{\textbf{\textit{#1}}}}
\usepackage{graphicx}
\makeatletter
\def\maxwidth{\ifdim\Gin@nat@width>\linewidth\linewidth\else\Gin@nat@width\fi}
\def\maxheight{\ifdim\Gin@nat@height>\textheight\textheight\else\Gin@nat@height\fi}
\makeatother
% Scale images if necessary, so that they will not overflow the page
% margins by default, and it is still possible to overwrite the defaults
% using explicit options in \includegraphics[width, height, ...]{}
\setkeys{Gin}{width=\maxwidth,height=\maxheight,keepaspectratio}
% Set default figure placement to htbp
\makeatletter
\def\fps@figure{htbp}
\makeatother
\setlength{\emergencystretch}{3em} % prevent overfull lines
\providecommand{\tightlist}{%
  \setlength{\itemsep}{0pt}\setlength{\parskip}{0pt}}
\setcounter{secnumdepth}{-\maxdimen} % remove section numbering
\ifluatex
  \usepackage{selnolig}  % disable illegal ligatures
\fi
\usepackage[]{natbib}
\bibliographystyle{plainnat}

\title{Nest Provisioning in a Fire Disturbed Landscape}
\author{Eliza Stein}
\date{11/8/2020}

\begin{document}
\maketitle

\hypertarget{introduction}{%
\section{Introduction}\label{introduction}}

Fire plays an important role as a consistent disturbance in maintaining
open stands of old-growth Ponderosa Pine (\emph{Pinus ponderosa})
forests by helping to eliminate understory and limit fuel loads
\citep{veblen2000climatic}. Before human intervention, Ponderosa Pine
forests naturally underwent forest fires in 5-50 year intervals
\citep{veblen2000climatic}. Over the past century, however, tree
planting initiatives and increased implementation of fire suppression
have led to increased density of stands \citep{griffis2001understory},
making forest stands that are already drought stressed even more
susceptible to high severity crown fires \citep{veblen2000climatic}. In
2002, a human-caused wildfire, the Hayman Fire, burned 138,000 acres of
old-growth Ponderosa pine forests in Colorado's Pike National Forest
\citep{graham2003hayman}.

The Flammulated Owl (\emph{Psiloscops flammeolus}) is a territorial,
insectivorous, and nocturnal raptor native to montane forests in
portions of the Rocky Mountains, Sierra Nevada Mountains, and the
Occidental Mountains \citep{linkhart2013flammulated}. The diet of the
owl primarily consists of moths native to these regions
\citep{linkhart2013flammulated}. As a highly specialized secondary
cavity nesting raptor, the Flammulated Owl is deemed an indicator
species, meaning that the health of an ecosystem can be estimated based
on the health of their population. Survival models have shown that
Flammulated Owl survival in the HFSA is currently lower than survival in
MGSA, suggesting that mortality, rather than emigration, explains most
of the population declines following the Hayman Fire (Linkhart and
Yanco, unpublished data).

Here, I examine one possible explanation for increased mortality in
HFSA: prey availability. High severity burns dramatically alter
vegetation structure, which in turn alters insect communities. Over
time, insect communities within high intensity burn scars can crash,
leaving avian predators without important food resources
\citep{nappi2010effect}. If Flammulated Owls are adapting their behavior
in response to changing prey availability, I would expect that the rate
of prey deliveries to active nests would increase or decrease (increase
if prey is lower quality, decrease if prey is more scarce or difficult
to detect) \citep{zarybnicka2009tengmalm}. If Flammulated Owls are not
adapting their behavior, this could mean that prey availability has
either not changed or, more likely, that Flammulated Owls, which do not
occupy landscapes prone to high severity burns, do not adapt their
behavior in response to large-scale landscape changes.

\hypertarget{initialization}{%
\section{Initialization}\label{initialization}}

All relative paths begin at the final-project repository root directory.

\hypertarget{required-packages}{%
\subsection{Required Packages}\label{required-packages}}

\begin{Shaded}
\begin{Highlighting}[]
\CommentTok{\# LaTex}
\FunctionTok{install.packages}\NormalTok{(}\StringTok{"tinytex"}\NormalTok{)}
\end{Highlighting}
\end{Shaded}

\begin{verbatim}
## Installing package into '/home/eliza/R/x86_64-pc-linux-gnu-library/4.0'
## (as 'lib' is unspecified)
\end{verbatim}

\begin{Shaded}
\begin{Highlighting}[]
\CommentTok{\# Data manipulation and visualization}
\FunctionTok{install.packages}\NormalTok{(}\StringTok{"tidyverse"}\NormalTok{, }\AttributeTok{repos=}\StringTok{\textquotesingle{}http://cran.us.r{-}project.org\textquotesingle{}}\NormalTok{)}
\end{Highlighting}
\end{Shaded}

\begin{verbatim}
## Installing package into '/home/eliza/R/x86_64-pc-linux-gnu-library/4.0'
## (as 'lib' is unspecified)
\end{verbatim}

\begin{Shaded}
\begin{Highlighting}[]
\FunctionTok{install.packages}\NormalTok{(}\StringTok{"gridExtra"}\NormalTok{, }\AttributeTok{repos=}\StringTok{\textquotesingle{}http://cran.us.r{-}project.org\textquotesingle{}}\NormalTok{)}
\end{Highlighting}
\end{Shaded}

\begin{verbatim}
## Installing package into '/home/eliza/R/x86_64-pc-linux-gnu-library/4.0'
## (as 'lib' is unspecified)
\end{verbatim}

\begin{Shaded}
\begin{Highlighting}[]
\CommentTok{\# LaTex}
\FunctionTok{library}\NormalTok{(tinytex)}
\NormalTok{tinytex}\SpecialCharTok{::}\FunctionTok{install\_tinytex}\NormalTok{()}
\end{Highlighting}
\end{Shaded}

\begin{verbatim}
## tlmgr option sys_bin ~/bin
\end{verbatim}

\begin{Shaded}
\begin{Highlighting}[]
\CommentTok{\# Data manipulation and visualization}
\FunctionTok{library}\NormalTok{(tidyverse)}
\end{Highlighting}
\end{Shaded}

\begin{verbatim}
## -- Attaching packages --------------------------------------- tidyverse 1.3.0 --
\end{verbatim}

\begin{verbatim}
## v ggplot2 3.3.2     v purrr   0.3.4
## v tibble  3.0.4     v dplyr   1.0.2
## v tidyr   1.1.2     v stringr 1.4.0
## v readr   1.4.0     v forcats 0.5.0
\end{verbatim}

\begin{verbatim}
## -- Conflicts ------------------------------------------ tidyverse_conflicts() --
## x dplyr::filter() masks stats::filter()
## x dplyr::lag()    masks stats::lag()
\end{verbatim}

\begin{Shaded}
\begin{Highlighting}[]
\FunctionTok{library}\NormalTok{(gridExtra)}
\end{Highlighting}
\end{Shaded}

\begin{verbatim}
## 
## Attaching package: 'gridExtra'
\end{verbatim}

\begin{verbatim}
## The following object is masked from 'package:dplyr':
## 
##     combine
\end{verbatim}

\hypertarget{working-directory}{%
\subsection{Working directory:}\label{working-directory}}

\begin{Shaded}
\begin{Highlighting}[]
\FunctionTok{setwd}\NormalTok{(}\StringTok{"\textasciitilde{}/final{-}project"}\NormalTok{)}
\end{Highlighting}
\end{Shaded}

\hypertarget{custom-functions}{%
\subsection{Custom functions:}\label{custom-functions}}

\begin{Shaded}
\begin{Highlighting}[]
\CommentTok{\#\textquotesingle{} getCI}
\CommentTok{\#\textquotesingle{} }
\CommentTok{\#\textquotesingle{} @param vec a vector}
\CommentTok{\#\textquotesingle{} @param n\_samp number of times to sample data}
\CommentTok{\#\textquotesingle{}}
\CommentTok{\#\textquotesingle{} @return upper and lower bootstrap confidence intervals}
\CommentTok{\#\textquotesingle{} }
\CommentTok{\#\textquotesingle{}}
\CommentTok{\#\textquotesingle{} @examples}
\CommentTok{\#\textquotesingle{}    getCI(1:20, 2000)}
\CommentTok{\#\textquotesingle{} @export}

\NormalTok{getCI }\OtherTok{\textless{}{-}} \ControlFlowTok{function}\NormalTok{(vec, }\AttributeTok{n\_samp=}\DecValTok{1000}\NormalTok{) \{}
\NormalTok{  smp }\OtherTok{\textless{}{-}} \FunctionTok{replicate}\NormalTok{(n\_samp, }\FunctionTok{mean}\NormalTok{(}\FunctionTok{sample}\NormalTok{(vec, }\AttributeTok{replace =} \ConstantTok{TRUE}\NormalTok{), }\AttributeTok{na.rm =} \ConstantTok{TRUE}\NormalTok{))}
\NormalTok{  CIs }\OtherTok{\textless{}{-}}\FunctionTok{quantile}\NormalTok{(smp, }\FunctionTok{c}\NormalTok{(}\FloatTok{0.025}\NormalTok{, }\FloatTok{0.975}\NormalTok{), }\AttributeTok{na.rm =}\NormalTok{ T)}
  \FunctionTok{return}\NormalTok{(CIs)}
\NormalTok{\}}
\end{Highlighting}
\end{Shaded}

\hypertarget{study-area}{%
\section{Study Area}\label{study-area}}

\hypertarget{hayman-fire-study-area-hfsa}{%
\subsection{Hayman Fire Study Area
(HFSA)}\label{hayman-fire-study-area-hfsa}}

Load in fire scar polygon. Projected coordinate reference system: UTM
Zone 13N.

Load in fire severity raster data:

Plot nest locations (n = 45) on Hayman Fire severity map:

Plot Hayman over CO basemap (maybe remove):

\hypertarget{prey-delivery-rates-by-sex}{%
\section{Prey Delivery Rates by Sex}\label{prey-delivery-rates-by-sex}}

Load prey delivery data.

\begin{Shaded}
\begin{Highlighting}[]
\NormalTok{pdOriginal }\OtherTok{\textless{}{-}} \FunctionTok{read.csv}\NormalTok{(}\StringTok{"../data/pd\_main.csv"}\NormalTok{)}


\CommentTok{\#rename the first column, which imported with a special character}
\FunctionTok{names}\NormalTok{(pdOriginal)[}\DecValTok{1}\NormalTok{] }\OtherTok{\textless{}{-}} \StringTok{"nest"}
\end{Highlighting}
\end{Shaded}

Filter to only include M and F (remove unknown and total), separate
``nest'' column into "study\_site' and territory.

\begin{Shaded}
\begin{Highlighting}[]
\NormalTok{pdMF }\OtherTok{\textless{}{-}}\NormalTok{ pdOriginal }\SpecialCharTok{\%\textgreater{}\%}
  \FunctionTok{separate}\NormalTok{(}\AttributeTok{col =}\NormalTok{ nest, }\AttributeTok{into =} \FunctionTok{c}\NormalTok{(}\StringTok{"study\_site"}\NormalTok{, }\StringTok{"territory"}\NormalTok{), }\AttributeTok{sep =} \DecValTok{1}\NormalTok{, }\AttributeTok{remove =} \ConstantTok{TRUE}\NormalTok{) }\SpecialCharTok{\%\textgreater{}\%}
  \FunctionTok{filter}\NormalTok{(sex }\SpecialCharTok{==} \StringTok{"M"} \SpecialCharTok{|}\NormalTok{ sex }\SpecialCharTok{==} \StringTok{"F"}\NormalTok{)}
\end{Highlighting}
\end{Shaded}

Check structure of data.

\begin{Shaded}
\begin{Highlighting}[]
\NormalTok{pd\_str }\OtherTok{\textless{}{-}} \FunctionTok{str}\NormalTok{(pdMF)}
\end{Highlighting}
\end{Shaded}

\begin{verbatim}
## 'data.frame':    1299 obs. of  55 variables:
##  $ study_site       : chr  "A" "A" "B" "C" ...
##  $ territory        : chr  "29_2007" "29_2007" "7_2005" "S1_1_2005" ...
##  $ year             : int  2007 2007 2005 2005 2005 2005 2006 2006 2014 2004 ...
##  $ obs_date         : chr  "6/4/2007" "6/4/2007" "6/15/2005" "6/9/2005" ...
##  $ clutch_size      : chr  "2" "2" "" "" ...
##  $ brood_size       : chr  "2" "2" "" "" ...
##  $ num_fledged      : chr  "2" "2" "" "" ...
##  $ incubation_start : chr  "6/6/2007" "6/6/2007" "" "" ...
##  $ julian_incubation: int  157 157 NA NA NA NA 157 157 155 151 ...
##  $ nest_age         : chr  "1" "1" "1" "1" ...
##  $ sunset           : int  2021 2021 NA 2025 2023 2020 2023 2023 2026 2019 ...
##  $ sex              : chr  "M" "F" "F" "F" ...
##  $ t15              : int  NA NA 0 NA NA NA NA NA 2 NA ...
##  $ t30              : int  NA NA 0 NA NA NA NA NA NA 0 ...
##  $ t45              : int  NA NA 0 NA 0 NA NA NA NA 2 ...
##  $ t60              : int  NA NA 0 NA 0 NA NA NA NA 1 ...
##  $ t75              : int  NA NA 0 NA 0 NA 1 0 NA 3 ...
##  $ t90              : int  NA NA NA 0 0 NA 0 0 NA 0 ...
##  $ t105             : int  NA NA NA 0 0 0 0 0 NA 0 ...
##  $ t120             : int  NA NA NA 0 NA 0 0 0 NA 0 ...
##  $ t135             : int  NA NA NA 0 NA 0 0 0 NA NA ...
##  $ t150             : int  NA NA NA 0 NA 0 1 0 NA NA ...
##  $ t165             : int  NA NA NA NA NA 0 0 0 NA NA ...
##  $ t180             : chr  "1" "0" "" "" ...
##  $ t195             : int  3 0 NA NA NA NA NA NA NA NA ...
##  $ t210             : int  2 0 NA NA NA NA NA NA NA NA ...
##  $ t225             : chr  "1" "0" "" "" ...
##  $ t240             : int  3 0 NA NA NA NA NA NA NA NA ...
##  $ t255             : int  NA NA NA NA NA NA NA NA NA NA ...
##  $ t270             : int  NA NA NA NA NA NA NA NA NA NA ...
##  $ t285             : int  NA NA NA NA NA NA NA NA NA NA ...
##  $ t300             : int  NA NA NA NA NA NA NA NA NA NA ...
##  $ t315             : int  NA NA NA NA NA NA NA NA NA NA ...
##  $ t330             : int  NA NA NA NA NA NA NA NA NA NA ...
##  $ t345             : int  NA NA NA NA NA NA NA NA NA NA ...
##  $ t360             : int  NA NA NA NA NA NA NA NA NA NA ...
##  $ t375             : int  NA NA NA NA NA NA NA NA NA NA ...
##  $ t390             : int  NA NA NA NA NA NA NA NA NA NA ...
##  $ t405             : int  NA NA NA NA NA NA NA NA NA NA ...
##  $ t420             : int  NA NA NA NA NA NA NA NA NA NA ...
##  $ t435             : int  NA NA NA NA NA NA NA NA NA NA ...
##  $ t450             : int  NA NA NA NA NA NA NA NA NA NA ...
##  $ t465             : int  NA NA NA NA NA NA NA NA NA NA ...
##  $ t480             : int  NA NA NA NA NA NA NA NA NA NA ...
##  $ t495             : int  NA NA NA NA NA NA NA NA NA NA ...
##  $ t510             : int  NA NA NA NA NA NA NA NA NA NA ...
##  $ t525             : int  NA NA NA NA NA NA NA NA NA NA ...
##  $ t540             : int  NA NA NA NA NA NA NA NA NA NA ...
##  $ Comments         : chr  "passive" "passive" "Captured on Last Pd" "Capture F, start time reflects when she goes back on, then capture attempt on M, F begging from last PD till end" ...
##  $ obs_time         : chr  "2313-2418" "2313-2418" "2015-2128" "2150-2254" ...
##  $ start_time       : chr  "23:13" "23:13" "20:15" "21:50" ...
##  $ stop_time        : chr  "24:18:00" "24:18:00" "21:28" "22:54" ...
##  $ weather          : chr  "sprinkles, light wind" "sprinkles, light wind" "storm entering around 2105, but doesn't start until after observations" "periods of showers" ...
##  $ fledge_date      : chr  "" "" "" "" ...
##  $ fledge_accuracy  : chr  "w/in 1 day" "w/in 1 day" "Predated" "abandoned" ...
\end{verbatim}

\begin{Shaded}
\begin{Highlighting}[]
\FunctionTok{unique}\NormalTok{(pdMF}\SpecialCharTok{$}\NormalTok{t180) }\CommentTok{\#at least one cell has an asterisk after the value}
\end{Highlighting}
\end{Shaded}

\begin{verbatim}
##  [1] "1"  "0"  ""   "4"  "4*" "2"  "9"  "10" "3"  "5"
\end{verbatim}

\begin{Shaded}
\begin{Highlighting}[]
\FunctionTok{unique}\NormalTok{(pdMF}\SpecialCharTok{$}\NormalTok{t225) }\CommentTok{\#same here}
\end{Highlighting}
\end{Shaded}

\begin{verbatim}
## [1] "1"  "0"  ""   "6*" "3"  "2"  "8"  "11" "7"
\end{verbatim}

\begin{Shaded}
\begin{Highlighting}[]
\FunctionTok{unique}\NormalTok{(pdMF}\SpecialCharTok{$}\NormalTok{nest\_age) }\CommentTok{\#"pred" and "" can be converted to NA}
\end{Highlighting}
\end{Shaded}

\begin{verbatim}
##  [1] "1"    "2"    "3"    "4"    "5"    "6"    "7"    "8"    "9"    "10"  
## [11] "11"   "12"   "13"   "14"   "15"   "16"   "17"   "18"   "19"   "20"  
## [21] "21"   "22"   "23"   "24"   "25"   "26"   "27"   "28"   "29"   "30"  
## [31] "31"   "32"   "33"   "34"   "35"   "36"   "37"   "38"   "39"   "40"  
## [41] "41"   "42"   "43"   "44"   "45"   "46"   "47"   "48"   "49"   "50"  
## [51] "51"   "53"   "100"  "0"    "pred" ""
\end{verbatim}

Fix structure.

\begin{Shaded}
\begin{Highlighting}[]
\CommentTok{\#remove asterisks}
\NormalTok{pdClean }\OtherTok{\textless{}{-}}\NormalTok{ pdMF }\SpecialCharTok{\%\textgreater{}\%}
  \FunctionTok{mutate}\NormalTok{(}\AttributeTok{t180 =} \FunctionTok{gsub}\NormalTok{(}\StringTok{"}\SpecialCharTok{\textbackslash{}\textbackslash{}}\StringTok{*"}\NormalTok{, }\StringTok{""}\NormalTok{, t180)) }\SpecialCharTok{\%\textgreater{}\%}
  \FunctionTok{mutate}\NormalTok{(}\AttributeTok{t225 =} \FunctionTok{gsub}\NormalTok{(}\StringTok{"}\SpecialCharTok{\textbackslash{}\textbackslash{}}\StringTok{*"}\NormalTok{, }\StringTok{""}\NormalTok{, t225))}


\CommentTok{\#change these columns to numeric}
\NormalTok{pdClean}\SpecialCharTok{$}\NormalTok{t180 }\OtherTok{\textless{}{-}} \FunctionTok{as.integer}\NormalTok{(pdClean}\SpecialCharTok{$}\NormalTok{t180)}
\NormalTok{pdClean}\SpecialCharTok{$}\NormalTok{t225 }\OtherTok{\textless{}{-}} \FunctionTok{as.integer}\NormalTok{(pdClean}\SpecialCharTok{$}\NormalTok{t225)}
\NormalTok{pdClean}\SpecialCharTok{$}\NormalTok{nest\_age }\OtherTok{\textless{}{-}} \FunctionTok{as.integer}\NormalTok{(pdClean}\SpecialCharTok{$}\NormalTok{nest\_age) }
\end{Highlighting}
\end{Shaded}

\begin{verbatim}
## Warning: NAs introduced by coercion
\end{verbatim}

\begin{Shaded}
\begin{Highlighting}[]
\CommentTok{\#warning here is ok{-}{-}NAs are replacing "pred" and "" values}
\end{Highlighting}
\end{Shaded}

\hypertarget{prey-deliveries-throughout-night}{%
\subsection{Prey Deliveries Throughout
Night}\label{prey-deliveries-throughout-night}}

Data was separated by sex (M vs.~F) and by incubation vs.~nestling
stage. Nestling period is defined as nest\_age \textgreater= 22 days. If
nest age was not indicated in original dataset, field notes were used to
determine whether nest was in incubation (all eggs) or nestling (at
least one nestling) stage. For these records, the following values were
manually input: nest\_age = 0 for incubation or nest\_age = 100 for
nestling, so that this data could be easily separated from known nest
age. If it was later determined that nest had been predated before
observation, ``pred'' was entered. If the nest stage could not be
determined, it was left blank.``pred'' and "" values were converted to
NA earlier when this column was converted to numeric.

\begin{Shaded}
\begin{Highlighting}[]
\CommentTok{\# Create independent dfs for M (nestling and incubation stage) and F (nestling and incubation state). }

\CommentTok{\#select relevant columns, add column for stage, rename study sites, drop NAs}
\NormalTok{pdStage1 }\OtherTok{\textless{}{-}}\NormalTok{ pdClean }\SpecialCharTok{\%\textgreater{}\%}
\NormalTok{  dplyr}\SpecialCharTok{::}\FunctionTok{select}\NormalTok{(sex, nest\_age, t15}\SpecialCharTok{:}\NormalTok{t240) }\SpecialCharTok{\%\textgreater{}\%}
  \FunctionTok{mutate}\NormalTok{(}
    \AttributeTok{stage =}
      \FunctionTok{ifelse}\NormalTok{(nest\_age }\SpecialCharTok{\textless{}} \DecValTok{22}\NormalTok{, }\StringTok{"incubation"}\NormalTok{, }\StringTok{"nestling"}\NormalTok{)) }\SpecialCharTok{\%\textgreater{}\%}
  \FunctionTok{drop\_na}\NormalTok{(stage)}

\CommentTok{\#change column names to remove "t" in front of time interval}
\FunctionTok{colnames}\NormalTok{(pdStage1) }\OtherTok{\textless{}{-}} \FunctionTok{c}\NormalTok{(}\StringTok{"sex"}\NormalTok{, }\StringTok{"nest\_age"}\NormalTok{, }\StringTok{"15"}\NormalTok{, }\StringTok{"30"}\NormalTok{, }\StringTok{"45"}\NormalTok{, }\StringTok{"60"}\NormalTok{, }\StringTok{"75"}\NormalTok{, }\StringTok{"90"}\NormalTok{, }\StringTok{"105"}\NormalTok{, }\StringTok{"120"}\NormalTok{, }\StringTok{"135"}\NormalTok{, }\StringTok{"150"}\NormalTok{, }\StringTok{"165"}\NormalTok{, }\StringTok{"180"}\NormalTok{, }\StringTok{"195"}\NormalTok{, }\StringTok{"210"}\NormalTok{, }\StringTok{"225"}\NormalTok{, }\StringTok{"240"}\NormalTok{, }\StringTok{"stage"}\NormalTok{)}

\CommentTok{\#create independent dfs for each study site and stage}
\NormalTok{pdM\_inc }\OtherTok{\textless{}{-}}\NormalTok{ pdStage1 }\SpecialCharTok{\%\textgreater{}\%}
  \FunctionTok{filter}\NormalTok{(sex }\SpecialCharTok{==}\StringTok{"M"}\NormalTok{, stage }\SpecialCharTok{==} \StringTok{"incubation"}\NormalTok{) }\SpecialCharTok{\%\textgreater{}\%}
\NormalTok{  dplyr}\SpecialCharTok{::}\FunctionTok{select}\NormalTok{(}\StringTok{\textquotesingle{}15\textquotesingle{}}\SpecialCharTok{:}\StringTok{\textquotesingle{}240\textquotesingle{}}\NormalTok{)}

\NormalTok{pdM\_nest }\OtherTok{\textless{}{-}}\NormalTok{ pdStage1 }\SpecialCharTok{\%\textgreater{}\%} 
  \FunctionTok{filter}\NormalTok{(sex }\SpecialCharTok{==}\StringTok{"M"}\NormalTok{, stage }\SpecialCharTok{==} \StringTok{"nestling"}\NormalTok{) }\SpecialCharTok{\%\textgreater{}\%}
\NormalTok{  dplyr}\SpecialCharTok{::}\FunctionTok{select}\NormalTok{(}\StringTok{\textquotesingle{}15\textquotesingle{}}\SpecialCharTok{:}\StringTok{\textquotesingle{}240\textquotesingle{}}\NormalTok{)}

\NormalTok{pdF\_inc }\OtherTok{\textless{}{-}}\NormalTok{ pdStage1 }\SpecialCharTok{\%\textgreater{}\%}
  \FunctionTok{filter}\NormalTok{(sex }\SpecialCharTok{==}\StringTok{"F"}\NormalTok{, stage }\SpecialCharTok{==} \StringTok{"incubation"}\NormalTok{) }\SpecialCharTok{\%\textgreater{}\%}
\NormalTok{  dplyr}\SpecialCharTok{::}\FunctionTok{select}\NormalTok{(}\StringTok{\textquotesingle{}15\textquotesingle{}}\SpecialCharTok{:}\StringTok{\textquotesingle{}240\textquotesingle{}}\NormalTok{)}
         
\NormalTok{pdF\_nest }\OtherTok{\textless{}{-}}\NormalTok{ pdStage1 }\SpecialCharTok{\%\textgreater{}\%}
  \FunctionTok{filter}\NormalTok{(sex }\SpecialCharTok{==}\StringTok{"F"}\NormalTok{, stage }\SpecialCharTok{==} \StringTok{"nestling"}\NormalTok{) }\SpecialCharTok{\%\textgreater{}\%}
\NormalTok{  dplyr}\SpecialCharTok{::}\FunctionTok{select}\NormalTok{(}\StringTok{\textquotesingle{}15\textquotesingle{}}\SpecialCharTok{:}\StringTok{\textquotesingle{}240\textquotesingle{}}\NormalTok{)}
\end{Highlighting}
\end{Shaded}

\hypertarget{mean-pd-tables}{%
\subsection{Mean PD tables}\label{mean-pd-tables}}

Four stand-alone data.frames, one for M (incubation), one for M
(nestling), one for F (incubation), and one for F (nestling):

\begin{Shaded}
\begin{Highlighting}[]
\NormalTok{meanM\_inc }\OtherTok{\textless{}{-}} \FunctionTok{data.frame}\NormalTok{(}
  \AttributeTok{time =} \FunctionTok{as.numeric}\NormalTok{(}\FunctionTok{colnames}\NormalTok{(pdM\_inc)),}
  \AttributeTok{M\_incubation =} \FunctionTok{colMeans}\NormalTok{(pdM\_inc, }\AttributeTok{na.rm =} \ConstantTok{TRUE}\NormalTok{))}

\NormalTok{meanM\_nest }\OtherTok{\textless{}{-}} \FunctionTok{data.frame}\NormalTok{(}
  \AttributeTok{time =} \FunctionTok{as.numeric}\NormalTok{(}\FunctionTok{colnames}\NormalTok{(pdM\_nest)),}
  \AttributeTok{M\_nestling =} \FunctionTok{colMeans}\NormalTok{(pdM\_nest, }\AttributeTok{na.rm =} \ConstantTok{TRUE}\NormalTok{))}

\NormalTok{meanF\_inc }\OtherTok{\textless{}{-}} \FunctionTok{data.frame}\NormalTok{(}
  \AttributeTok{time =} \FunctionTok{as.numeric}\NormalTok{(}\FunctionTok{colnames}\NormalTok{(pdF\_inc)),}
  \AttributeTok{F\_incubation =} \FunctionTok{colMeans}\NormalTok{(pdF\_inc, }\AttributeTok{na.rm =} \ConstantTok{TRUE}\NormalTok{))}

\NormalTok{meanF\_nest }\OtherTok{\textless{}{-}} \FunctionTok{data.frame}\NormalTok{(}
  \AttributeTok{time =} \FunctionTok{as.numeric}\NormalTok{(}\FunctionTok{colnames}\NormalTok{(pdF\_nest)),}
  \AttributeTok{F\_nestling =} \FunctionTok{colMeans}\NormalTok{(pdF\_nest, }\AttributeTok{na.rm =} \ConstantTok{TRUE}\NormalTok{))}
\end{Highlighting}
\end{Shaded}

One table showing mean PDs for all study sites and stages:

\begin{Shaded}
\begin{Highlighting}[]
\NormalTok{meanMF }\OtherTok{\textless{}{-}} \FunctionTok{data.frame}\NormalTok{(}
  \AttributeTok{time =} \FunctionTok{as.numeric}\NormalTok{(}\FunctionTok{colnames}\NormalTok{(pdM\_inc)),}
  \AttributeTok{M\_incubation =} \FunctionTok{colMeans}\NormalTok{(pdM\_inc, }\AttributeTok{na.rm =} \ConstantTok{TRUE}\NormalTok{),}
  \AttributeTok{M\_nestling =} \FunctionTok{colMeans}\NormalTok{(pdM\_nest, }\AttributeTok{na.rm =} \ConstantTok{TRUE}\NormalTok{),}
  \AttributeTok{F\_incubation =} \FunctionTok{colMeans}\NormalTok{(pdF\_inc, }\AttributeTok{na.rm =} \ConstantTok{TRUE}\NormalTok{),}
  \AttributeTok{F\_nestling =} \FunctionTok{colMeans}\NormalTok{(pdF\_nest, }\AttributeTok{na.rm =} \ConstantTok{TRUE}\NormalTok{)}
\NormalTok{)}
\end{Highlighting}
\end{Shaded}

\hypertarget{calculate-confidence-intervals}{%
\subsection{Calculate confidence
intervals}\label{calculate-confidence-intervals}}

Write and apply function to obtain CIs.

\begin{Shaded}
\begin{Highlighting}[]
\CommentTok{\#now apply it across the columns of PD data}
\NormalTok{ciM\_inc }\OtherTok{\textless{}{-}} \FunctionTok{apply}\NormalTok{(pdM\_inc, }\DecValTok{2}\NormalTok{, }\AttributeTok{FUN =}\NormalTok{ getCI)}
\NormalTok{ciM\_nest }\OtherTok{\textless{}{-}} \FunctionTok{apply}\NormalTok{(pdM\_nest, }\DecValTok{2}\NormalTok{, }\AttributeTok{FUN =}\NormalTok{ getCI)}
\NormalTok{ciF\_inc }\OtherTok{\textless{}{-}} \FunctionTok{apply}\NormalTok{(pdF\_inc, }\DecValTok{2}\NormalTok{, }\AttributeTok{FUN =}\NormalTok{ getCI)}
\NormalTok{ciF\_nest }\OtherTok{\textless{}{-}} \FunctionTok{apply}\NormalTok{(pdF\_nest, }\DecValTok{2}\NormalTok{, }\AttributeTok{FUN =}\NormalTok{ getCI)}
\end{Highlighting}
\end{Shaded}

Add CIs to a data frame. Remove rows where time \textgreater{} 180 for
ciInc because no data is available for MGSA after this time.

\begin{Shaded}
\begin{Highlighting}[]
\NormalTok{ciInc\_sex }\OtherTok{\textless{}{-}} \FunctionTok{data.frame}\NormalTok{(}
    \AttributeTok{sex =} \FunctionTok{c}\NormalTok{(}\FunctionTok{rep}\NormalTok{(}\StringTok{"M"}\NormalTok{, }\FunctionTok{nrow}\NormalTok{(meanM\_inc)), }\FunctionTok{rep}\NormalTok{(}\StringTok{"F"}\NormalTok{, }\FunctionTok{nrow}\NormalTok{(meanF\_inc))),}
    \AttributeTok{mean =} \FunctionTok{c}\NormalTok{(}\FunctionTok{colMeans}\NormalTok{(pdM\_inc, }\AttributeTok{na.rm =} \ConstantTok{TRUE}\NormalTok{), }\FunctionTok{colMeans}\NormalTok{(pdF\_inc, }\AttributeTok{na.rm =} \ConstantTok{TRUE}\NormalTok{)),}
    \AttributeTok{ci\_l =} \FunctionTok{c}\NormalTok{(ciM\_inc[}\DecValTok{1}\NormalTok{,], ciF\_inc[}\DecValTok{1}\NormalTok{,]),}
    \AttributeTok{ci\_h =} \FunctionTok{c}\NormalTok{(ciM\_inc[}\DecValTok{2}\NormalTok{,], ciF\_inc[}\DecValTok{2}\NormalTok{,]),}
    \AttributeTok{time =} \FunctionTok{c}\NormalTok{(}\FunctionTok{as.numeric}\NormalTok{(}\FunctionTok{rownames}\NormalTok{(meanM\_inc)), }\FunctionTok{as.numeric}\NormalTok{(}\FunctionTok{rownames}\NormalTok{(meanF\_inc))),}
    \AttributeTok{stage =} \StringTok{"Incubation"}\NormalTok{)}


\NormalTok{ciNest\_sex }\OtherTok{\textless{}{-}} \FunctionTok{data.frame}\NormalTok{(}
    \AttributeTok{sex =} \FunctionTok{c}\NormalTok{(}\FunctionTok{rep}\NormalTok{(}\StringTok{"M"}\NormalTok{, }\FunctionTok{nrow}\NormalTok{(meanM\_nest)), }\FunctionTok{rep}\NormalTok{(}\StringTok{"F"}\NormalTok{, }\FunctionTok{nrow}\NormalTok{(meanF\_nest))),}
    \AttributeTok{mean =} \FunctionTok{c}\NormalTok{(}\FunctionTok{colMeans}\NormalTok{(pdM\_nest, }\AttributeTok{na.rm =} \ConstantTok{TRUE}\NormalTok{), }\FunctionTok{colMeans}\NormalTok{(pdF\_nest, }\AttributeTok{na.rm =} \ConstantTok{TRUE}\NormalTok{)),}
    \AttributeTok{ci\_l =} \FunctionTok{c}\NormalTok{(ciM\_nest[}\DecValTok{1}\NormalTok{,], ciF\_nest[}\DecValTok{1}\NormalTok{,]),}
    \AttributeTok{ci\_h =} \FunctionTok{c}\NormalTok{(ciM\_nest[}\DecValTok{2}\NormalTok{,], ciF\_nest[}\DecValTok{2}\NormalTok{,]),}
    \AttributeTok{time =} \FunctionTok{c}\NormalTok{(}\FunctionTok{as.numeric}\NormalTok{(}\FunctionTok{rownames}\NormalTok{(meanM\_nest)), }\FunctionTok{as.numeric}\NormalTok{(}\FunctionTok{rownames}\NormalTok{(meanF\_nest))),}
    \AttributeTok{stage =} \StringTok{"Nestling"}\NormalTok{)}
    
\NormalTok{ciAll\_sex }\OtherTok{\textless{}{-}} \FunctionTok{rbind}\NormalTok{(ciInc\_sex, ciNest\_sex)}
\end{Highlighting}
\end{Shaded}

\hypertarget{plot}{%
\subsection{Plot}\label{plot}}

Incubation:

\begin{Shaded}
\begin{Highlighting}[]
\NormalTok{plotInc\_sex }\OtherTok{\textless{}{-}} \FunctionTok{ggplot}\NormalTok{(}\AttributeTok{data =}\NormalTok{ ciInc\_sex) }\SpecialCharTok{+}
  \FunctionTok{geom\_point}\NormalTok{(}\FunctionTok{aes}\NormalTok{(}\AttributeTok{x =}\NormalTok{ time, }\AttributeTok{y =}\NormalTok{ mean, }\AttributeTok{color =}\NormalTok{ sex, }\AttributeTok{group =}\NormalTok{ sex),}
             \AttributeTok{position =} \FunctionTok{position\_dodge}\NormalTok{(}\AttributeTok{width=}\FloatTok{0.75}\NormalTok{)) }\SpecialCharTok{+}
  \FunctionTok{geom\_errorbar}\NormalTok{(}\FunctionTok{aes}\NormalTok{(}\AttributeTok{x=}\NormalTok{time, }\AttributeTok{ymax =}\NormalTok{ ci\_h, }\AttributeTok{ymin=}\NormalTok{ci\_l, }\AttributeTok{color =}\NormalTok{ sex, }
                    \AttributeTok{group =}\NormalTok{ sex),}
                \AttributeTok{position =} \FunctionTok{position\_dodge}\NormalTok{(}\AttributeTok{width=}\FloatTok{0.75}\NormalTok{)) }\SpecialCharTok{+}
  \FunctionTok{labs}\NormalTok{(}\AttributeTok{x =} \StringTok{"Time After Sunset (minutes)"}\NormalTok{, }\AttributeTok{y =} \StringTok{"Mean Prey Deliveries"}\NormalTok{, }
       \AttributeTok{title =} \StringTok{"Incubation"}\NormalTok{, }\AttributeTok{color =} \StringTok{"Sex"}\NormalTok{) }\SpecialCharTok{+}
  \FunctionTok{theme\_minimal}\NormalTok{() }\SpecialCharTok{+}
  \FunctionTok{theme}\NormalTok{(}\AttributeTok{plot.title =} \FunctionTok{element\_text}\NormalTok{(}\AttributeTok{hjust =} \FloatTok{0.5}\NormalTok{))}
\end{Highlighting}
\end{Shaded}

Nestling:

\begin{Shaded}
\begin{Highlighting}[]
\NormalTok{plotNest\_sex }\OtherTok{\textless{}{-}} \FunctionTok{ggplot}\NormalTok{(}\AttributeTok{data =}\NormalTok{ ciNest\_sex) }\SpecialCharTok{+}
  \FunctionTok{geom\_point}\NormalTok{(}\FunctionTok{aes}\NormalTok{(}\AttributeTok{x =}\NormalTok{ time, }\AttributeTok{y =}\NormalTok{ mean, }\AttributeTok{color =}\NormalTok{ sex, }\AttributeTok{group =}\NormalTok{ sex),}
             \AttributeTok{position =} \FunctionTok{position\_dodge}\NormalTok{(}\AttributeTok{width=}\FloatTok{0.75}\NormalTok{)) }\SpecialCharTok{+}
  \FunctionTok{geom\_errorbar}\NormalTok{(}\FunctionTok{aes}\NormalTok{(}\AttributeTok{x=}\NormalTok{time, }\AttributeTok{ymax =}\NormalTok{ ci\_h, }\AttributeTok{ymin=}\NormalTok{ci\_l, }\AttributeTok{color =}\NormalTok{ sex, }
                    \AttributeTok{group =}\NormalTok{ sex),}
                \AttributeTok{position =} \FunctionTok{position\_dodge}\NormalTok{(}\AttributeTok{width=}\FloatTok{0.75}\NormalTok{)) }\SpecialCharTok{+}
  \FunctionTok{labs}\NormalTok{(}\AttributeTok{x =} \StringTok{"Time After Sunset (minutes)"}\NormalTok{, }\AttributeTok{y =} \StringTok{"Mean Prey Deliveries"}\NormalTok{, }
       \AttributeTok{title =} \StringTok{"Nestling"}\NormalTok{, }\AttributeTok{color =} \StringTok{"Sex"}\NormalTok{) }\SpecialCharTok{+}
  \FunctionTok{theme\_minimal}\NormalTok{() }\SpecialCharTok{+}
  \FunctionTok{theme}\NormalTok{(}\AttributeTok{plot.title =} \FunctionTok{element\_text}\NormalTok{(}\AttributeTok{hjust =} \FloatTok{0.5}\NormalTok{))}
\end{Highlighting}
\end{Shaded}

Plot both as separate plots, side by side:

\begin{Shaded}
\begin{Highlighting}[]
\FunctionTok{grid.arrange}\NormalTok{(plotInc\_sex, plotNest\_sex)}
\end{Highlighting}
\end{Shaded}

\includegraphics{pd_analysis_files/figure-latex/unnamed-chunk-22-1.pdf}

\hypertarget{test-for-difference}{%
\subsection{Test for Difference}\label{test-for-difference}}

Independent t-test for incubation stage. p = 5.163e-09

\begin{Shaded}
\begin{Highlighting}[]
\FunctionTok{t.test}\NormalTok{(meanF\_inc}\SpecialCharTok{$}\NormalTok{F\_incubation, meanM\_inc}\SpecialCharTok{$}\NormalTok{M\_incubation)}
\end{Highlighting}
\end{Shaded}

\begin{verbatim}
## 
##  Welch Two Sample t-test
## 
## data:  meanF_inc$F_incubation and meanM_inc$M_incubation
## t = -11.795, df = 15.078, p-value = 5.163e-09
## alternative hypothesis: true difference in means is not equal to 0
## 95 percent confidence interval:
##  -1.1125110 -0.7720969
## sample estimates:
##   mean of x   mean of y 
## 0.004058442 0.946362390
\end{verbatim}

Independent t-test for nestling stage. p = 1.355e-08

\begin{Shaded}
\begin{Highlighting}[]
\FunctionTok{t.test}\NormalTok{(meanF\_nest}\SpecialCharTok{$}\NormalTok{F\_nestling, meanM\_nest}\SpecialCharTok{$}\NormalTok{M\_nestling)}
\end{Highlighting}
\end{Shaded}

\begin{verbatim}
## 
##  Welch Two Sample t-test
## 
## data:  meanF_nest$F_nestling and meanM_nest$M_nestling
## t = -8.0503, df = 26.534, p-value = 1.355e-08
## alternative hypothesis: true difference in means is not equal to 0
## 95 percent confidence interval:
##  -1.2074419 -0.7166369
## sample estimates:
## mean of x mean of y 
## 0.3543829 1.3164223
\end{verbatim}

\hypertarget{prey-delivery-rates-by-site}{%
\section{Prey Delivery Rates by
Site}\label{prey-delivery-rates-by-site}}

Filter pdMF to only include males in B (MGSA) and C (HFSA).

\begin{Shaded}
\begin{Highlighting}[]
\NormalTok{pdHM }\OtherTok{\textless{}{-}}\NormalTok{ pdClean }\SpecialCharTok{\%\textgreater{}\%}
  \FunctionTok{filter}\NormalTok{(study\_site }\SpecialCharTok{==} \StringTok{"B"} \SpecialCharTok{|}\NormalTok{ study\_site }\SpecialCharTok{==} \StringTok{"C"}\NormalTok{, sex }\SpecialCharTok{==} \StringTok{"M"}\NormalTok{)}
\end{Highlighting}
\end{Shaded}

\hypertarget{visualize-prey-deliveries-throughout-night}{%
\subsection{Visualize prey deliveries throughout
night}\label{visualize-prey-deliveries-throughout-night}}

Data was separated by study site (HFSA vs.~MGSA) and by incubation
vs.~nestling stage.

since the last observation for HFSA is at time = 240, we'll end both
datasets there.

First, create data frame with PDs for whole dataset, then broken down by
each study site and stage.

\begin{Shaded}
\begin{Highlighting}[]
\CommentTok{\#create independet dfs for HFSA (nestling and incubation stage) and MGSA (nestling and incubation state). }

\CommentTok{\#select relevant columns, add column for stage, rename study sites, drop NAs}
\NormalTok{pdStage }\OtherTok{\textless{}{-}}\NormalTok{ pdClean }\SpecialCharTok{\%\textgreater{}\%}
\NormalTok{  dplyr}\SpecialCharTok{::}\FunctionTok{select}\NormalTok{(study\_site, nest\_age, t15}\SpecialCharTok{:}\NormalTok{t240) }\SpecialCharTok{\%\textgreater{}\%}
  \FunctionTok{mutate}\NormalTok{(}
    \AttributeTok{stage =}
      \FunctionTok{ifelse}\NormalTok{(nest\_age }\SpecialCharTok{\textless{}} \DecValTok{22}\NormalTok{, }\StringTok{"incubation"}\NormalTok{, }\StringTok{"nestling"}\NormalTok{),}
    \AttributeTok{study\_site =}
      \FunctionTok{ifelse}\NormalTok{(study\_site }\SpecialCharTok{==} \StringTok{"B"}\NormalTok{, }\StringTok{"MGSA"}\NormalTok{, }\StringTok{"HFSA"}\NormalTok{)) }\SpecialCharTok{\%\textgreater{}\%}
  \FunctionTok{drop\_na}\NormalTok{(stage)}

\CommentTok{\#change column names to remove "t" in front of time interval}
\FunctionTok{colnames}\NormalTok{(pdStage) }\OtherTok{\textless{}{-}} \FunctionTok{c}\NormalTok{(}\StringTok{"study\_site"}\NormalTok{, }\StringTok{"nest\_age"}\NormalTok{, }\StringTok{"15"}\NormalTok{, }\StringTok{"30"}\NormalTok{, }\StringTok{"45"}\NormalTok{, }\StringTok{"60"}\NormalTok{, }\StringTok{"75"}\NormalTok{, }\StringTok{"90"}\NormalTok{, }\StringTok{"105"}\NormalTok{, }\StringTok{"120"}\NormalTok{, }\StringTok{"135"}\NormalTok{, }\StringTok{"150"}\NormalTok{, }\StringTok{"165"}\NormalTok{, }\StringTok{"180"}\NormalTok{, }\StringTok{"195"}\NormalTok{, }\StringTok{"210"}\NormalTok{, }\StringTok{"225"}\NormalTok{, }\StringTok{"240"}\NormalTok{, }\StringTok{"stage"}\NormalTok{)}

\CommentTok{\#create independent dfs for each study site and stage}
\NormalTok{pdHFSA\_inc }\OtherTok{\textless{}{-}}\NormalTok{ pdStage }\SpecialCharTok{\%\textgreater{}\%}
  \FunctionTok{filter}\NormalTok{(study\_site }\SpecialCharTok{==}\StringTok{"HFSA"}\NormalTok{, stage }\SpecialCharTok{==} \StringTok{"incubation"}\NormalTok{) }\SpecialCharTok{\%\textgreater{}\%}
\NormalTok{  dplyr}\SpecialCharTok{::}\FunctionTok{select}\NormalTok{(}\StringTok{\textquotesingle{}15\textquotesingle{}}\SpecialCharTok{:}\StringTok{\textquotesingle{}240\textquotesingle{}}\NormalTok{)}

\NormalTok{pdHFSA\_nest }\OtherTok{\textless{}{-}}\NormalTok{ pdStage }\SpecialCharTok{\%\textgreater{}\%} 
  \FunctionTok{filter}\NormalTok{(study\_site }\SpecialCharTok{==}\StringTok{"HFSA"}\NormalTok{, stage }\SpecialCharTok{==} \StringTok{"nestling"}\NormalTok{) }\SpecialCharTok{\%\textgreater{}\%}
\NormalTok{  dplyr}\SpecialCharTok{::}\FunctionTok{select}\NormalTok{(}\StringTok{\textquotesingle{}15\textquotesingle{}}\SpecialCharTok{:}\StringTok{\textquotesingle{}240\textquotesingle{}}\NormalTok{)}

\NormalTok{pdMGSA\_inc }\OtherTok{\textless{}{-}}\NormalTok{ pdStage }\SpecialCharTok{\%\textgreater{}\%}
  \FunctionTok{filter}\NormalTok{(study\_site }\SpecialCharTok{==}\StringTok{"MGSA"}\NormalTok{, stage }\SpecialCharTok{==} \StringTok{"incubation"}\NormalTok{) }\SpecialCharTok{\%\textgreater{}\%}
\NormalTok{  dplyr}\SpecialCharTok{::}\FunctionTok{select}\NormalTok{(}\StringTok{\textquotesingle{}15\textquotesingle{}}\SpecialCharTok{:}\StringTok{\textquotesingle{}240\textquotesingle{}}\NormalTok{)}
         
\NormalTok{pdMGSA\_nest }\OtherTok{\textless{}{-}}\NormalTok{ pdStage }\SpecialCharTok{\%\textgreater{}\%}
  \FunctionTok{filter}\NormalTok{(study\_site }\SpecialCharTok{==}\StringTok{"MGSA"}\NormalTok{, stage }\SpecialCharTok{==} \StringTok{"nestling"}\NormalTok{) }\SpecialCharTok{\%\textgreater{}\%}
\NormalTok{  dplyr}\SpecialCharTok{::}\FunctionTok{select}\NormalTok{(}\StringTok{\textquotesingle{}15\textquotesingle{}}\SpecialCharTok{:}\StringTok{\textquotesingle{}240\textquotesingle{}}\NormalTok{)}
\end{Highlighting}
\end{Shaded}

\hypertarget{mean-pd-tables-1}{%
\subsection{Mean PD tables}\label{mean-pd-tables-1}}

Four stand-alone data.frames, one for (incubation), one for HFSA
(nestling), one for MGSA(incubation), and one for MGSA (nestling):

\begin{Shaded}
\begin{Highlighting}[]
\NormalTok{meanHFSA\_inc }\OtherTok{\textless{}{-}} \FunctionTok{data.frame}\NormalTok{(}
  \AttributeTok{time =} \FunctionTok{as.numeric}\NormalTok{(}\FunctionTok{colnames}\NormalTok{(pdHFSA\_inc)),}
  \AttributeTok{HFSA\_incubation =} \FunctionTok{colMeans}\NormalTok{(pdHFSA\_inc, }\AttributeTok{na.rm =} \ConstantTok{TRUE}\NormalTok{))}

\NormalTok{meanHFSA\_nest }\OtherTok{\textless{}{-}} \FunctionTok{data.frame}\NormalTok{(}
  \AttributeTok{time =} \FunctionTok{as.numeric}\NormalTok{(}\FunctionTok{colnames}\NormalTok{(pdHFSA\_nest)),}
  \AttributeTok{HFSA\_nestling =} \FunctionTok{colMeans}\NormalTok{(pdHFSA\_nest, }\AttributeTok{na.rm =} \ConstantTok{TRUE}\NormalTok{))}

\NormalTok{meanMGSA\_inc }\OtherTok{\textless{}{-}} \FunctionTok{data.frame}\NormalTok{(}
  \AttributeTok{time =} \FunctionTok{as.numeric}\NormalTok{(}\FunctionTok{colnames}\NormalTok{(pdMGSA\_inc)),}
  \AttributeTok{MGSA\_incubation =} \FunctionTok{colMeans}\NormalTok{(pdMGSA\_inc, }\AttributeTok{na.rm =} \ConstantTok{TRUE}\NormalTok{))}

\NormalTok{meanMGSA\_nest }\OtherTok{\textless{}{-}} \FunctionTok{data.frame}\NormalTok{(}
  \AttributeTok{time =} \FunctionTok{as.numeric}\NormalTok{(}\FunctionTok{colnames}\NormalTok{(pdMGSA\_nest)),}
  \AttributeTok{MGSA\_nestling =} \FunctionTok{colMeans}\NormalTok{(pdMGSA\_nest, }\AttributeTok{na.rm =} \ConstantTok{TRUE}\NormalTok{))}
\end{Highlighting}
\end{Shaded}

Two tables, one for nestling stage and one for incubation stage. This is
just for visualization, not used in further analysis.

One table showing mean PDs for all study sites and stages:

\begin{Shaded}
\begin{Highlighting}[]
\NormalTok{meanAll }\OtherTok{\textless{}{-}} \FunctionTok{data.frame}\NormalTok{(}
  \AttributeTok{time =} \FunctionTok{as.numeric}\NormalTok{(}\FunctionTok{colnames}\NormalTok{(pdHFSA\_inc)),}
  \AttributeTok{HFSA\_incubation =} \FunctionTok{colMeans}\NormalTok{(pdHFSA\_inc, }\AttributeTok{na.rm =} \ConstantTok{TRUE}\NormalTok{),}
  \AttributeTok{HFSA\_nestling =} \FunctionTok{colMeans}\NormalTok{(pdHFSA\_nest, }\AttributeTok{na.rm =} \ConstantTok{TRUE}\NormalTok{),}
  \AttributeTok{MGSA\_incubation =} \FunctionTok{colMeans}\NormalTok{(pdMGSA\_inc, }\AttributeTok{na.rm =} \ConstantTok{TRUE}\NormalTok{),}
  \AttributeTok{MGSA\_nestling =} \FunctionTok{colMeans}\NormalTok{(pdMGSA\_nest, }\AttributeTok{na.rm =} \ConstantTok{TRUE}\NormalTok{)}
\NormalTok{)}
\end{Highlighting}
\end{Shaded}

\hypertarget{calculate-confidence-intervals-1}{%
\subsection{Calculate confidence
intervals}\label{calculate-confidence-intervals-1}}

Write and apply function to obtain CIs.

\begin{Shaded}
\begin{Highlighting}[]
\CommentTok{\#now apply it across the columns of PD data}
\NormalTok{ciHFSA\_inc }\OtherTok{\textless{}{-}} \FunctionTok{apply}\NormalTok{(pdHFSA\_inc, }\DecValTok{2}\NormalTok{, }\AttributeTok{FUN =}\NormalTok{ getCI)}
\NormalTok{ciHFSA\_nest }\OtherTok{\textless{}{-}} \FunctionTok{apply}\NormalTok{(pdHFSA\_nest, }\DecValTok{2}\NormalTok{, }\AttributeTok{FUN =}\NormalTok{ getCI)}
\NormalTok{ciMGSA\_inc }\OtherTok{\textless{}{-}} \FunctionTok{apply}\NormalTok{(pdMGSA\_inc, }\DecValTok{2}\NormalTok{, }\AttributeTok{FUN =}\NormalTok{ getCI)}
\NormalTok{ciMGSA\_nest }\OtherTok{\textless{}{-}} \FunctionTok{apply}\NormalTok{(pdMGSA\_nest, }\DecValTok{2}\NormalTok{, }\AttributeTok{FUN =}\NormalTok{ getCI)}
\end{Highlighting}
\end{Shaded}

Add CIs to a data frame. Remove rows where time \textgreater{} 180 for
ciInc because no data is available for MGSA after this time.

\begin{Shaded}
\begin{Highlighting}[]
\NormalTok{ciInc }\OtherTok{\textless{}{-}} \FunctionTok{filter}\NormalTok{(}
  \FunctionTok{data.frame}\NormalTok{(}
    \AttributeTok{study\_area =} \FunctionTok{c}\NormalTok{(}\FunctionTok{rep}\NormalTok{(}\StringTok{"HFSA"}\NormalTok{, }\FunctionTok{nrow}\NormalTok{(meanHFSA\_inc)), }\FunctionTok{rep}\NormalTok{(}\StringTok{"MGSA"}\NormalTok{, }\FunctionTok{nrow}\NormalTok{(meanMGSA\_inc))),}
    \AttributeTok{mean =} \FunctionTok{c}\NormalTok{(}\FunctionTok{colMeans}\NormalTok{(pdHFSA\_inc, }\AttributeTok{na.rm =} \ConstantTok{TRUE}\NormalTok{), }\FunctionTok{colMeans}\NormalTok{(pdMGSA\_inc, }\AttributeTok{na.rm =} \ConstantTok{TRUE}\NormalTok{)),}
    \AttributeTok{ci\_l =} \FunctionTok{c}\NormalTok{(ciHFSA\_inc[}\DecValTok{1}\NormalTok{,], ciMGSA\_inc[}\DecValTok{1}\NormalTok{,]),}
    \AttributeTok{ci\_h =} \FunctionTok{c}\NormalTok{(ciHFSA\_inc[}\DecValTok{2}\NormalTok{,], ciMGSA\_inc[}\DecValTok{2}\NormalTok{,]),}
    \AttributeTok{time =} \FunctionTok{c}\NormalTok{(}\FunctionTok{as.numeric}\NormalTok{(}\FunctionTok{rownames}\NormalTok{(meanHFSA\_inc)), }\FunctionTok{as.numeric}\NormalTok{(}\FunctionTok{rownames}\NormalTok{(meanMGSA\_inc))),}
    \AttributeTok{stage =} \StringTok{"Incubation"}\NormalTok{),}
\NormalTok{  time }\SpecialCharTok{\textless{}=} \DecValTok{180}\NormalTok{)}


\NormalTok{ciNest }\OtherTok{\textless{}{-}} \FunctionTok{data.frame}\NormalTok{(}
    \AttributeTok{study\_area =} \FunctionTok{c}\NormalTok{(}\FunctionTok{rep}\NormalTok{(}\StringTok{"HFSA"}\NormalTok{, }\FunctionTok{nrow}\NormalTok{(meanHFSA\_nest)), }\FunctionTok{rep}\NormalTok{(}\StringTok{"MGSA"}\NormalTok{, }\FunctionTok{nrow}\NormalTok{(meanMGSA\_nest))),}
    \AttributeTok{mean =} \FunctionTok{c}\NormalTok{(}\FunctionTok{colMeans}\NormalTok{(pdHFSA\_nest, }\AttributeTok{na.rm =} \ConstantTok{TRUE}\NormalTok{), }\FunctionTok{colMeans}\NormalTok{(pdMGSA\_nest, }\AttributeTok{na.rm =} \ConstantTok{TRUE}\NormalTok{)),}
    \AttributeTok{ci\_l =} \FunctionTok{c}\NormalTok{(ciHFSA\_nest[}\DecValTok{1}\NormalTok{,], ciMGSA\_nest[}\DecValTok{1}\NormalTok{,]),}
    \AttributeTok{ci\_h =} \FunctionTok{c}\NormalTok{(ciHFSA\_nest[}\DecValTok{2}\NormalTok{,], ciMGSA\_nest[}\DecValTok{2}\NormalTok{,]),}
    \AttributeTok{time =} \FunctionTok{c}\NormalTok{(}\FunctionTok{as.numeric}\NormalTok{(}\FunctionTok{rownames}\NormalTok{(meanHFSA\_nest)), }\FunctionTok{as.numeric}\NormalTok{(}\FunctionTok{rownames}\NormalTok{(meanMGSA\_nest))),}
    \AttributeTok{stage =} \StringTok{"Nestling"}\NormalTok{)}
    
\NormalTok{ciAll }\OtherTok{\textless{}{-}} \FunctionTok{rbind}\NormalTok{(ciInc, ciNest)}
\end{Highlighting}
\end{Shaded}

\hypertarget{plot-1}{%
\subsection{Plot}\label{plot-1}}

Incubation:

\begin{Shaded}
\begin{Highlighting}[]
\NormalTok{plotInc }\OtherTok{\textless{}{-}} \FunctionTok{ggplot}\NormalTok{(}\AttributeTok{data =}\NormalTok{ ciInc) }\SpecialCharTok{+}
  \FunctionTok{geom\_point}\NormalTok{(}\FunctionTok{aes}\NormalTok{(}\AttributeTok{x =}\NormalTok{ time, }\AttributeTok{y =}\NormalTok{ mean, }\AttributeTok{color =}\NormalTok{ study\_area, }\AttributeTok{group =}\NormalTok{ study\_area),}
             \AttributeTok{position =} \FunctionTok{position\_dodge}\NormalTok{(}\AttributeTok{width=}\FloatTok{0.75}\NormalTok{)) }\SpecialCharTok{+}
  \FunctionTok{geom\_errorbar}\NormalTok{(}\FunctionTok{aes}\NormalTok{(}\AttributeTok{x=}\NormalTok{time, }\AttributeTok{ymax =}\NormalTok{ ci\_h, }\AttributeTok{ymin=}\NormalTok{ci\_l, }\AttributeTok{color =}\NormalTok{ study\_area, }
                    \AttributeTok{group =}\NormalTok{ study\_area),}
                \AttributeTok{position =} \FunctionTok{position\_dodge}\NormalTok{(}\AttributeTok{width=}\FloatTok{0.75}\NormalTok{)) }\SpecialCharTok{+}
  \FunctionTok{labs}\NormalTok{(}\AttributeTok{x =} \StringTok{"Time After Sunset (minutes)"}\NormalTok{, }\AttributeTok{y =} \StringTok{"Mean Prey Deliveries"}\NormalTok{, }
       \AttributeTok{title =} \StringTok{"Incubation"}\NormalTok{, }\AttributeTok{color =} \StringTok{"Study Area"}\NormalTok{) }\SpecialCharTok{+}
  \FunctionTok{theme\_minimal}\NormalTok{() }\SpecialCharTok{+}
  \FunctionTok{theme}\NormalTok{(}\AttributeTok{plot.title =} \FunctionTok{element\_text}\NormalTok{(}\AttributeTok{hjust =} \FloatTok{0.5}\NormalTok{))}
\end{Highlighting}
\end{Shaded}

Nestling:

\begin{Shaded}
\begin{Highlighting}[]
\NormalTok{plotNest }\OtherTok{\textless{}{-}} \FunctionTok{ggplot}\NormalTok{(}\AttributeTok{data =}\NormalTok{ ciNest) }\SpecialCharTok{+}
  \FunctionTok{geom\_point}\NormalTok{(}\FunctionTok{aes}\NormalTok{(}\AttributeTok{x =}\NormalTok{ time, }\AttributeTok{y =}\NormalTok{ mean, }\AttributeTok{color =}\NormalTok{ study\_area, }\AttributeTok{group =}\NormalTok{ study\_area),}
             \AttributeTok{position =} \FunctionTok{position\_dodge}\NormalTok{(}\AttributeTok{width=}\FloatTok{0.75}\NormalTok{)) }\SpecialCharTok{+}
  \FunctionTok{geom\_errorbar}\NormalTok{(}\FunctionTok{aes}\NormalTok{(}\AttributeTok{x=}\NormalTok{time, }\AttributeTok{ymax =}\NormalTok{ ci\_h, }\AttributeTok{ymin=}\NormalTok{ci\_l, }\AttributeTok{color =}\NormalTok{ study\_area, }
                    \AttributeTok{group =}\NormalTok{ study\_area),}
                \AttributeTok{position =} \FunctionTok{position\_dodge}\NormalTok{(}\AttributeTok{width=}\FloatTok{0.75}\NormalTok{)) }\SpecialCharTok{+}
  \FunctionTok{labs}\NormalTok{(}\AttributeTok{x =} \StringTok{"Time After Sunset (minutes)"}\NormalTok{, }\AttributeTok{y =} \StringTok{"Mean Prey Deliveries"}\NormalTok{, }
       \AttributeTok{title =} \StringTok{"Nestling"}\NormalTok{, }\AttributeTok{color =} \StringTok{"Study Area"}\NormalTok{) }\SpecialCharTok{+}
  \FunctionTok{theme\_minimal}\NormalTok{() }\SpecialCharTok{+}
  \FunctionTok{theme}\NormalTok{(}\AttributeTok{plot.title =} \FunctionTok{element\_text}\NormalTok{(}\AttributeTok{hjust =} \FloatTok{0.5}\NormalTok{))}
\end{Highlighting}
\end{Shaded}

Both:

\begin{Shaded}
\begin{Highlighting}[]
\NormalTok{plotAll }\OtherTok{\textless{}{-}} \FunctionTok{ggplot}\NormalTok{(}\AttributeTok{data =}\NormalTok{ ciAll) }\SpecialCharTok{+}
  \FunctionTok{geom\_point}\NormalTok{(}\FunctionTok{aes}\NormalTok{(}\AttributeTok{x =}\NormalTok{ time, }\AttributeTok{y =}\NormalTok{ mean, }\AttributeTok{color =}\NormalTok{ study\_area, }\AttributeTok{group =}\NormalTok{ study\_area),}
             \AttributeTok{position =} \FunctionTok{position\_dodge}\NormalTok{(}\AttributeTok{width=}\FloatTok{0.75}\NormalTok{)) }\SpecialCharTok{+}
  \FunctionTok{geom\_errorbar}\NormalTok{(}\FunctionTok{aes}\NormalTok{(}\AttributeTok{x=}\NormalTok{time, }\AttributeTok{ymax =}\NormalTok{ ci\_h, }\AttributeTok{ymin=}\NormalTok{ci\_l, }\AttributeTok{color =}\NormalTok{ study\_area, }
                    \AttributeTok{group =}\NormalTok{ study\_area),}
                \AttributeTok{position =} \FunctionTok{position\_dodge}\NormalTok{(}\AttributeTok{width=}\FloatTok{0.75}\NormalTok{)) }\SpecialCharTok{+}
  \FunctionTok{labs}\NormalTok{(}\AttributeTok{x =} \StringTok{"Time After Sunset (minutes)"}\NormalTok{, }\AttributeTok{y =} \StringTok{"Mean Prey Deliveries"}\NormalTok{, }
       \AttributeTok{title =} \StringTok{"Average Prey Deliveries Throughout Night"}\NormalTok{, }\AttributeTok{color =} \StringTok{"Study Area"}\NormalTok{) }\SpecialCharTok{+}
  \FunctionTok{theme\_minimal}\NormalTok{() }\SpecialCharTok{+}
  \FunctionTok{theme}\NormalTok{(}\AttributeTok{plot.title =} \FunctionTok{element\_text}\NormalTok{(}\AttributeTok{hjust =} \FloatTok{0.5}\NormalTok{)) }\SpecialCharTok{+}
  \FunctionTok{facet\_grid}\NormalTok{(}\SpecialCharTok{\textasciitilde{}}\NormalTok{ stage)}
\end{Highlighting}
\end{Shaded}

Plot both as separate plots, side by side:

\begin{Shaded}
\begin{Highlighting}[]
\FunctionTok{grid.arrange}\NormalTok{(plotInc, plotNest)}
\end{Highlighting}
\end{Shaded}

\includegraphics{pd_analysis_files/figure-latex/unnamed-chunk-34-1.pdf}

\hypertarget{test-for-difference-1}{%
\subsection{Test for Difference}\label{test-for-difference-1}}

Independent t-tests for incubation stage. p = 0.8786

\begin{Shaded}
\begin{Highlighting}[]
\FunctionTok{t.test}\NormalTok{(meanHFSA\_inc}\SpecialCharTok{$}\NormalTok{HFSA\_incubation, meanMGSA\_inc}\SpecialCharTok{$}\NormalTok{MGSA\_incubation)}
\end{Highlighting}
\end{Shaded}

\begin{verbatim}
## 
##  Welch Two Sample t-test
## 
## data:  meanHFSA_inc$HFSA_incubation and meanMGSA_inc$MGSA_incubation
## t = 0.4698, df = 21.664, p-value = 0.6432
## alternative hypothesis: true difference in means is not equal to 0
## 95 percent confidence interval:
##  -0.1252531  0.1985364
## sample estimates:
## mean of x mean of y 
## 0.4873974 0.4507558
\end{verbatim}

Independent t-tests for nestling stage. p = 0.377

\begin{Shaded}
\begin{Highlighting}[]
\FunctionTok{t.test}\NormalTok{(meanHFSA\_nest}\SpecialCharTok{$}\NormalTok{HFSA\_nestling, meanMGSA\_nest}\SpecialCharTok{$}\NormalTok{MGSA\_nestling)}
\end{Highlighting}
\end{Shaded}

\begin{verbatim}
## 
##  Welch Two Sample t-test
## 
## data:  meanHFSA_nest$HFSA_nestling and meanMGSA_nest$MGSA_nestling
## t = 0.89919, df = 25.484, p-value = 0.377
## alternative hypothesis: true difference in means is not equal to 0
## 95 percent confidence interval:
##  -0.1472872  0.3759510
## sample estimates:
## mean of x mean of y 
## 0.9076640 0.7933321
\end{verbatim}

  \bibliography{references.bib}

\end{document}
