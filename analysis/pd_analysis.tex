% Options for packages loaded elsewhere
\PassOptionsToPackage{unicode}{hyperref}
\PassOptionsToPackage{hyphens}{url}
%
\documentclass[
]{article}
\usepackage{lmodern}
\usepackage{amsmath}
\usepackage{ifxetex,ifluatex}
\ifnum 0\ifxetex 1\fi\ifluatex 1\fi=0 % if pdftex
  \usepackage[T1]{fontenc}
  \usepackage[utf8]{inputenc}
  \usepackage{textcomp} % provide euro and other symbols
  \usepackage{amssymb}
\else % if luatex or xetex
  \usepackage{unicode-math}
  \defaultfontfeatures{Scale=MatchLowercase}
  \defaultfontfeatures[\rmfamily]{Ligatures=TeX,Scale=1}
\fi
% Use upquote if available, for straight quotes in verbatim environments
\IfFileExists{upquote.sty}{\usepackage{upquote}}{}
\IfFileExists{microtype.sty}{% use microtype if available
  \usepackage[]{microtype}
  \UseMicrotypeSet[protrusion]{basicmath} % disable protrusion for tt fonts
}{}
\makeatletter
\@ifundefined{KOMAClassName}{% if non-KOMA class
  \IfFileExists{parskip.sty}{%
    \usepackage{parskip}
  }{% else
    \setlength{\parindent}{0pt}
    \setlength{\parskip}{6pt plus 2pt minus 1pt}}
}{% if KOMA class
  \KOMAoptions{parskip=half}}
\makeatother
\usepackage{xcolor}
\IfFileExists{xurl.sty}{\usepackage{xurl}}{} % add URL line breaks if available
\IfFileExists{bookmark.sty}{\usepackage{bookmark}}{\usepackage{hyperref}}
\hypersetup{
  pdftitle={Nest Provisioning in a Fire Disturbed Landscape},
  pdfauthor={Eliza Stein},
  hidelinks,
  pdfcreator={LaTeX via pandoc}}
\urlstyle{same} % disable monospaced font for URLs
\usepackage[margin=1in]{geometry}
\usepackage{color}
\usepackage{fancyvrb}
\newcommand{\VerbBar}{|}
\newcommand{\VERB}{\Verb[commandchars=\\\{\}]}
\DefineVerbatimEnvironment{Highlighting}{Verbatim}{commandchars=\\\{\}}
% Add ',fontsize=\small' for more characters per line
\usepackage{framed}
\definecolor{shadecolor}{RGB}{248,248,248}
\newenvironment{Shaded}{\begin{snugshade}}{\end{snugshade}}
\newcommand{\AlertTok}[1]{\textcolor[rgb]{0.94,0.16,0.16}{#1}}
\newcommand{\AnnotationTok}[1]{\textcolor[rgb]{0.56,0.35,0.01}{\textbf{\textit{#1}}}}
\newcommand{\AttributeTok}[1]{\textcolor[rgb]{0.77,0.63,0.00}{#1}}
\newcommand{\BaseNTok}[1]{\textcolor[rgb]{0.00,0.00,0.81}{#1}}
\newcommand{\BuiltInTok}[1]{#1}
\newcommand{\CharTok}[1]{\textcolor[rgb]{0.31,0.60,0.02}{#1}}
\newcommand{\CommentTok}[1]{\textcolor[rgb]{0.56,0.35,0.01}{\textit{#1}}}
\newcommand{\CommentVarTok}[1]{\textcolor[rgb]{0.56,0.35,0.01}{\textbf{\textit{#1}}}}
\newcommand{\ConstantTok}[1]{\textcolor[rgb]{0.00,0.00,0.00}{#1}}
\newcommand{\ControlFlowTok}[1]{\textcolor[rgb]{0.13,0.29,0.53}{\textbf{#1}}}
\newcommand{\DataTypeTok}[1]{\textcolor[rgb]{0.13,0.29,0.53}{#1}}
\newcommand{\DecValTok}[1]{\textcolor[rgb]{0.00,0.00,0.81}{#1}}
\newcommand{\DocumentationTok}[1]{\textcolor[rgb]{0.56,0.35,0.01}{\textbf{\textit{#1}}}}
\newcommand{\ErrorTok}[1]{\textcolor[rgb]{0.64,0.00,0.00}{\textbf{#1}}}
\newcommand{\ExtensionTok}[1]{#1}
\newcommand{\FloatTok}[1]{\textcolor[rgb]{0.00,0.00,0.81}{#1}}
\newcommand{\FunctionTok}[1]{\textcolor[rgb]{0.00,0.00,0.00}{#1}}
\newcommand{\ImportTok}[1]{#1}
\newcommand{\InformationTok}[1]{\textcolor[rgb]{0.56,0.35,0.01}{\textbf{\textit{#1}}}}
\newcommand{\KeywordTok}[1]{\textcolor[rgb]{0.13,0.29,0.53}{\textbf{#1}}}
\newcommand{\NormalTok}[1]{#1}
\newcommand{\OperatorTok}[1]{\textcolor[rgb]{0.81,0.36,0.00}{\textbf{#1}}}
\newcommand{\OtherTok}[1]{\textcolor[rgb]{0.56,0.35,0.01}{#1}}
\newcommand{\PreprocessorTok}[1]{\textcolor[rgb]{0.56,0.35,0.01}{\textit{#1}}}
\newcommand{\RegionMarkerTok}[1]{#1}
\newcommand{\SpecialCharTok}[1]{\textcolor[rgb]{0.00,0.00,0.00}{#1}}
\newcommand{\SpecialStringTok}[1]{\textcolor[rgb]{0.31,0.60,0.02}{#1}}
\newcommand{\StringTok}[1]{\textcolor[rgb]{0.31,0.60,0.02}{#1}}
\newcommand{\VariableTok}[1]{\textcolor[rgb]{0.00,0.00,0.00}{#1}}
\newcommand{\VerbatimStringTok}[1]{\textcolor[rgb]{0.31,0.60,0.02}{#1}}
\newcommand{\WarningTok}[1]{\textcolor[rgb]{0.56,0.35,0.01}{\textbf{\textit{#1}}}}
\usepackage{graphicx}
\makeatletter
\def\maxwidth{\ifdim\Gin@nat@width>\linewidth\linewidth\else\Gin@nat@width\fi}
\def\maxheight{\ifdim\Gin@nat@height>\textheight\textheight\else\Gin@nat@height\fi}
\makeatother
% Scale images if necessary, so that they will not overflow the page
% margins by default, and it is still possible to overwrite the defaults
% using explicit options in \includegraphics[width, height, ...]{}
\setkeys{Gin}{width=\maxwidth,height=\maxheight,keepaspectratio}
% Set default figure placement to htbp
\makeatletter
\def\fps@figure{htbp}
\makeatother
\setlength{\emergencystretch}{3em} % prevent overfull lines
\providecommand{\tightlist}{%
  \setlength{\itemsep}{0pt}\setlength{\parskip}{0pt}}
\setcounter{secnumdepth}{-\maxdimen} % remove section numbering
\ifluatex
  \usepackage{selnolig}  % disable illegal ligatures
\fi
\usepackage[]{natbib}
\bibliographystyle{plainnat}

\title{Nest Provisioning in a Fire Disturbed Landscape}
\author{Eliza Stein}
\date{11/8/2020}

\begin{document}
\maketitle

\hypertarget{introduction}{%
\section{Introduction}\label{introduction}}

Fire plays an important role as a consistent disturbance in maintaining
open stands of old-growth Ponderosa Pine (\emph{Pinus ponderosa})
forests by helping to eliminate understory and limit fuel loads
\citep{veblen2000climatic}. Before human intervention, Ponderosa Pine
forests naturally underwent forest fires in 5-50 year intervals
\citep{veblen2000climatic}. Over the past century, however, tree
planting initiatives and increased implementation of fire suppression
have led to increased density of stands \citep{griffis2001understory},
making forest stands that are already drought stressed even more
susceptible to high severity crown fires \citep{veblen2000climatic}. In
2002, a human-caused wildfire, the Hayman Fire, burned 138,000 acres of
old-growth Ponderosa pine forests in Colorado's Pike National Forest
\citep{graham2003hayman}.

The Flammulated Owl (\emph{Psiloscops flammeolus}) is a territorial,
insectivorous, and nocturnal raptor native to montane forests in
portions of the Rocky Mountains, Sierra Nevada Mountains, and the
Occidental Mountains \citep{linkhart2013flammulated}. The diet of the
owl primarily consists of moths native to these regions
\citep{linkhart2013flammulated}. As a highly specialized secondary
cavity nesting raptor, the Flammulated Owl is deemed an indicator
species, meaning that the health of an ecosystem can be estimated based
on the health of their population. Survival models have shown that
Flammulated Owl survival in the HFSA is currently lower than survival in
MGSA, suggesting that mortality, rather than emigration, explains most
of the population declines following the Hayman Fire (Linkhart and
Yanco, unpublished data).

Here, I examine one possible explanation for increased mortality in
HFSA: prey availability. High severity burns dramatically alter
vegetation structure, which in turn alters insect communities. Over
time, insect communities within high intensity burn scars can crash,
leaving avian predators without important food resources
\citep{nappi2010effect}. If Flammulated Owls are adapting their behavior
in response to changing prey availability, I would expect that the rate
of prey deliveries to active nests would increase or decrease (increase
if prey is lower quality, decrease if prey is more scarce or difficult
to detect) \citep{zarybnicka2009tengmalm}. If Flammulated Owls are not
adapting their behavior, this could mean that prey availability has
either not changed or, more likely, that Flammulated Owls, which do not
occupy landscapes prone to high severity burns, do not adapt their
behavior in response to large-scale landscape changes. This would make
them highly sensitive to large-scale disturbances that affect foraging
at important times, such as during the breeding season.

The objectives for this analysis are to examine:

\begin{enumerate}
\def\labelenumi{\arabic{enumi}.}
\tightlist
\item
  How do prey delivery rates change throughout the night?
\item
  How do male and female nest provisioning differ?
\item
  How is nest provisioning affected by high severity burns?
\end{enumerate}

\hypertarget{methods}{%
\section{Methods}\label{methods}}

\hypertarget{data-collection}{%
\subsection{Data Collection}\label{data-collection}}

From summer 2002 to summer 2020, researchers monitored Flammulated Owl
territories in the Hayman Fire Study Area (HFSA), located in the western
portion of the Hayman Fire scar, and four other study areas within a 10
mile radius: Missouri Gulch Study Area (MGSA), Hotel Gulch Study Area
(HGSA), and Trout Creek Study Area (TCSA). Researchers detected all
Flammulated Owl nests at the beginning of each breeding season and
monitored then until fledging or predation occurred. Each nest was
observed for 1-2 hours per week, during which time researchers recorded
the number of times the male or female breeder delivered prey to the
nest. During incubation, the male exclusively delivers prey to the
female, who stays on the nest. During the nestling stage, both parents
delivery prey, although the female spends less time on each forage and
sits on the nest between prey deliveries (observational data). While
observing, researchers recorded which individual delivered prey, which
was determined by vocal cues if both individuals were off the nest at
the time. Data was recorded in fifteen minute intervals.

\hypertarget{analysis}{%
\subsection{Analysis}\label{analysis}}

First, average prey delivery rates for both males and females were
compared at 15-minute intervals throughout the night. Any prey
deliveries recorded at ``Unknown'' were discarded. Bootstrap confidence
intervals were generated by sampling each time interval 1000 times, and
means with CIs were visualized by plotting.

Then, the data was filtered to include only HFSA, our treatment area,
and MGSA, an unburned study site \textasciitilde5 miles from HFSA, with
similar habitat. Means and bootstrap confidence intervals were plotted
using the above technique.

\hypertarget{analysis-1}{%
\section{Analysis}\label{analysis-1}}

\hypertarget{initialization}{%
\section{Initialization}\label{initialization}}

All relative paths begin at the final-project/analysis subdirectory.

\begin{Shaded}
\begin{Highlighting}[]
\NormalTok{knitr}\SpecialCharTok{::}\NormalTok{opts\_chunk}\SpecialCharTok{$}\FunctionTok{set}\NormalTok{(}\AttributeTok{message =} \ConstantTok{FALSE}\NormalTok{)}
\end{Highlighting}
\end{Shaded}

\hypertarget{required-packages}{%
\subsection{Required Packages}\label{required-packages}}

\begin{Shaded}
\begin{Highlighting}[]
\CommentTok{\# Data manipulation and visualization}
\FunctionTok{library}\NormalTok{(tidyverse)}
\FunctionTok{library}\NormalTok{(gridExtra)}
\end{Highlighting}
\end{Shaded}

\hypertarget{working-directory}{%
\subsection{Working directory:}\label{working-directory}}

\hypertarget{custom-functions}{%
\subsection{Custom functions:}\label{custom-functions}}

getCI is designed to generate bootstrap confidence intervals from a
given vector.

\begin{Shaded}
\begin{Highlighting}[]
\CommentTok{\#\textquotesingle{} getCI}
\CommentTok{\#\textquotesingle{} }
\CommentTok{\#\textquotesingle{} @param vec a vector}
\CommentTok{\#\textquotesingle{} @param n\_samp number of times to sample data}
\CommentTok{\#\textquotesingle{}}
\CommentTok{\#\textquotesingle{} @return upper and lower bootstrap confidence intervals}
\CommentTok{\#\textquotesingle{} }
\CommentTok{\#\textquotesingle{}}
\CommentTok{\#\textquotesingle{} @examples}
\CommentTok{\#\textquotesingle{}    getCI(1:20, 2000)}
\CommentTok{\#\textquotesingle{} @export}

\NormalTok{getCI }\OtherTok{\textless{}{-}} \ControlFlowTok{function}\NormalTok{(vec, }\AttributeTok{n\_samp=}\DecValTok{1000}\NormalTok{) \{}
\NormalTok{  smp }\OtherTok{\textless{}{-}} \FunctionTok{replicate}\NormalTok{(n\_samp, }\FunctionTok{mean}\NormalTok{(}\FunctionTok{sample}\NormalTok{(vec, }\AttributeTok{replace =} \ConstantTok{TRUE}\NormalTok{), }\AttributeTok{na.rm =} \ConstantTok{TRUE}\NormalTok{))}
\NormalTok{  CIs }\OtherTok{\textless{}{-}}\FunctionTok{quantile}\NormalTok{(smp, }\FunctionTok{c}\NormalTok{(}\FloatTok{0.025}\NormalTok{, }\FloatTok{0.975}\NormalTok{), }\AttributeTok{na.rm =}\NormalTok{ T)}
  \FunctionTok{return}\NormalTok{(CIs)}
\NormalTok{\}}
\end{Highlighting}
\end{Shaded}

testNormality runs a Shapiro-Wilk test on each time interval in a pd
data frame and returns p values based on the null hypothesis that the
data is normally distributed. If p \textless{} 0.05, data is not
normally distributed.

\begin{Shaded}
\begin{Highlighting}[]
\CommentTok{\#\textquotesingle{} testNormality}
\CommentTok{\#\textquotesingle{} }
\CommentTok{\#\textquotesingle{} @param dat a data.frame where columns = time intervals and rows = \# prey deliveries}
\CommentTok{\#\textquotesingle{} @param margin.a margin for apply statement running shapiro.test. default = 2.}
\CommentTok{\#\textquotesingle{}}
\CommentTok{\#\textquotesingle{} @return vector of p.values for each column of dat}
\CommentTok{\#\textquotesingle{} }
\CommentTok{\#\textquotesingle{}}
\CommentTok{\#\textquotesingle{} @examples}
\CommentTok{\#\textquotesingle{}    testNormality(pdM\_inc, 2)}
\CommentTok{\#\textquotesingle{} @export}

\NormalTok{testNormality }\OtherTok{\textless{}{-}} \ControlFlowTok{function}\NormalTok{(dat, }\AttributeTok{margin =} \DecValTok{2}\NormalTok{) \{}
\NormalTok{  norms }\OtherTok{\textless{}{-}} \FunctionTok{apply}\NormalTok{(dat, margin, }\AttributeTok{FUN =}\NormalTok{ shapiro.test)}
  \FunctionTok{return}\NormalTok{(}\FunctionTok{sapply}\NormalTok{(norms, }\ControlFlowTok{function}\NormalTok{(x)\{x[[}\StringTok{"p.value"}\NormalTok{]]\}))}
\NormalTok{\}}
\end{Highlighting}
\end{Shaded}

testWilcox runs a Wilcoxon test on the nth columns of two different data
frames (i.e.~df1{[}1{]} and df2{[}1{]}, df1{[}2{]} and df2{[}2{]},
etc.). We add the argument exact = FALSE to suppress the Warning message
that exact p.values with ties can't be computed (warning comes from
assumption that values are continuous). This is useful in testing for
difference between prey delivery observations at time = 15 in one study
site and at time = 15 in a second study site, when the data are not
normally distributed.

\begin{Shaded}
\begin{Highlighting}[]
\CommentTok{\#\textquotesingle{} testWilcox}
\CommentTok{\#\textquotesingle{} }
\CommentTok{\#\textquotesingle{} @param a a data.frame where columns = time intervals and rows = \# prey deliveries}
\CommentTok{\#\textquotesingle{} @param b a second data.frame where columns = time intervals and rows = \# prey deliveries}
\CommentTok{\#\textquotesingle{}}
\CommentTok{\#\textquotesingle{} @return data.frame of p values (values) and corresponding time intervals (ind)}
\CommentTok{\#\textquotesingle{} }
\CommentTok{\#\textquotesingle{}}
\CommentTok{\#\textquotesingle{} @examples}
\CommentTok{\#\textquotesingle{}    testWilcox(pdMGSA\_inc, pdHFSA\_inc)}
\CommentTok{\#\textquotesingle{} @export}

\NormalTok{testWilcox }\OtherTok{\textless{}{-}} \ControlFlowTok{function}\NormalTok{(a, b)\{}
\NormalTok{  tmp }\OtherTok{\textless{}{-}} \FunctionTok{mapply}\NormalTok{(wilcox.test, a, b, }\AttributeTok{exact =} \ConstantTok{FALSE}\NormalTok{)}
\NormalTok{  p.values }\OtherTok{\textless{}{-}} \FunctionTok{stack}\NormalTok{(}\FunctionTok{mapply}\NormalTok{(}\ControlFlowTok{function}\NormalTok{(x, y) }\FunctionTok{wilcox.test}\NormalTok{(x, y, }\AttributeTok{exact =} \ConstantTok{FALSE}\NormalTok{)}\SpecialCharTok{$}\NormalTok{p.value, a, b))}
  \FunctionTok{return}\NormalTok{(p.values)}
\NormalTok{\}}
\end{Highlighting}
\end{Shaded}

\hypertarget{study-area}{%
\section{Study Area}\label{study-area}}

\hypertarget{hayman-fire-study-area-hfsa}{%
\subsection{Hayman Fire Study Area
(HFSA)}\label{hayman-fire-study-area-hfsa}}

Load in fire scar polygon. Projected coordinate reference system: UTM
Zone 13N.

Load in fire severity raster data:

Plot nest locations (n = 45) on Hayman Fire severity map:

Plot Hayman over CO basemap (maybe remove):

\hypertarget{prey-delivery-rates-by-sex}{%
\section{Prey Delivery Rates by Sex}\label{prey-delivery-rates-by-sex}}

First, we compared average prey delivery rates for males and females at
all study sties at fifteen minute intervals throughout the night.

\hypertarget{read-in-data}{%
\subsection{Read in Data}\label{read-in-data}}

Load prey delivery data.

\begin{Shaded}
\begin{Highlighting}[]
\NormalTok{pdOriginal }\OtherTok{\textless{}{-}} \FunctionTok{read.csv}\NormalTok{(}\StringTok{"../data/pd\_main.csv"}\NormalTok{)}

\CommentTok{\#rename the first column, which imported with a special character}
\FunctionTok{names}\NormalTok{(pdOriginal)[}\DecValTok{1}\NormalTok{] }\OtherTok{\textless{}{-}} \StringTok{"nest"}
\end{Highlighting}
\end{Shaded}

Filter to only include M and F (remove `unknown' and `total'), separate
`nest' column into `study\_site' and `territory.'

\begin{Shaded}
\begin{Highlighting}[]
\NormalTok{pdMF }\OtherTok{\textless{}{-}}\NormalTok{ pdOriginal }\SpecialCharTok{\%\textgreater{}\%}
  \FunctionTok{separate}\NormalTok{(}\AttributeTok{col =}\NormalTok{ nest, }\AttributeTok{into =} \FunctionTok{c}\NormalTok{(}\StringTok{"study\_site"}\NormalTok{, }\StringTok{"territory"}\NormalTok{), }\AttributeTok{sep =} \DecValTok{1}\NormalTok{, }\AttributeTok{remove =} \ConstantTok{TRUE}\NormalTok{) }\SpecialCharTok{\%\textgreater{}\%}
  \FunctionTok{filter}\NormalTok{(sex }\SpecialCharTok{==} \StringTok{"M"} \SpecialCharTok{|}\NormalTok{ sex }\SpecialCharTok{==} \StringTok{"F"}\NormalTok{)}
\end{Highlighting}
\end{Shaded}

Check structure of data.

\begin{Shaded}
\begin{Highlighting}[]
\NormalTok{pd\_str }\OtherTok{\textless{}{-}} \FunctionTok{str}\NormalTok{(pdMF)}
\end{Highlighting}
\end{Shaded}

\begin{verbatim}
## 'data.frame':    1299 obs. of  55 variables:
##  $ study_site       : chr  "A" "A" "B" "C" ...
##  $ territory        : chr  "29_2007" "29_2007" "7_2005" "S1_1_2005" ...
##  $ year             : int  2007 2007 2005 2005 2005 2005 2006 2006 2014 2004 ...
##  $ obs_date         : chr  "6/4/2007" "6/4/2007" "6/15/2005" "6/9/2005" ...
##  $ clutch_size      : chr  "2" "2" "" "" ...
##  $ brood_size       : chr  "2" "2" "" "" ...
##  $ num_fledged      : chr  "2" "2" "" "" ...
##  $ incubation_start : chr  "6/6/2007" "6/6/2007" "" "" ...
##  $ julian_incubation: int  157 157 NA NA NA NA 157 157 155 151 ...
##  $ nest_age         : chr  "1" "1" "1" "1" ...
##  $ sunset           : int  2021 2021 NA 2025 2023 2020 2023 2023 2026 2019 ...
##  $ sex              : chr  "M" "F" "F" "F" ...
##  $ t15              : int  NA NA 0 NA NA NA NA NA 2 NA ...
##  $ t30              : int  NA NA 0 NA NA NA NA NA NA 0 ...
##  $ t45              : int  NA NA 0 NA 0 NA NA NA NA 2 ...
##  $ t60              : int  NA NA 0 NA 0 NA NA NA NA 1 ...
##  $ t75              : int  NA NA 0 NA 0 NA 1 0 NA 3 ...
##  $ t90              : int  NA NA NA 0 0 NA 0 0 NA 0 ...
##  $ t105             : int  NA NA NA 0 0 0 0 0 NA 0 ...
##  $ t120             : int  NA NA NA 0 NA 0 0 0 NA 0 ...
##  $ t135             : int  NA NA NA 0 NA 0 0 0 NA NA ...
##  $ t150             : int  NA NA NA 0 NA 0 1 0 NA NA ...
##  $ t165             : int  NA NA NA NA NA 0 0 0 NA NA ...
##  $ t180             : chr  "1" "0" "" "" ...
##  $ t195             : int  3 0 NA NA NA NA NA NA NA NA ...
##  $ t210             : int  2 0 NA NA NA NA NA NA NA NA ...
##  $ t225             : chr  "1" "0" "" "" ...
##  $ t240             : int  3 0 NA NA NA NA NA NA NA NA ...
##  $ t255             : int  NA NA NA NA NA NA NA NA NA NA ...
##  $ t270             : int  NA NA NA NA NA NA NA NA NA NA ...
##  $ t285             : int  NA NA NA NA NA NA NA NA NA NA ...
##  $ t300             : int  NA NA NA NA NA NA NA NA NA NA ...
##  $ t315             : int  NA NA NA NA NA NA NA NA NA NA ...
##  $ t330             : int  NA NA NA NA NA NA NA NA NA NA ...
##  $ t345             : int  NA NA NA NA NA NA NA NA NA NA ...
##  $ t360             : int  NA NA NA NA NA NA NA NA NA NA ...
##  $ t375             : int  NA NA NA NA NA NA NA NA NA NA ...
##  $ t390             : int  NA NA NA NA NA NA NA NA NA NA ...
##  $ t405             : int  NA NA NA NA NA NA NA NA NA NA ...
##  $ t420             : int  NA NA NA NA NA NA NA NA NA NA ...
##  $ t435             : int  NA NA NA NA NA NA NA NA NA NA ...
##  $ t450             : int  NA NA NA NA NA NA NA NA NA NA ...
##  $ t465             : int  NA NA NA NA NA NA NA NA NA NA ...
##  $ t480             : int  NA NA NA NA NA NA NA NA NA NA ...
##  $ t495             : int  NA NA NA NA NA NA NA NA NA NA ...
##  $ t510             : int  NA NA NA NA NA NA NA NA NA NA ...
##  $ t525             : int  NA NA NA NA NA NA NA NA NA NA ...
##  $ t540             : int  NA NA NA NA NA NA NA NA NA NA ...
##  $ Comments         : chr  "passive" "passive" "Captured on Last Pd" "Capture F, start time reflects when she goes back on, then capture attempt on M, F begging from last PD till end" ...
##  $ obs_time         : chr  "2313-2418" "2313-2418" "2015-2128" "2150-2254" ...
##  $ start_time       : chr  "23:13" "23:13" "20:15" "21:50" ...
##  $ stop_time        : chr  "24:18:00" "24:18:00" "21:28" "22:54" ...
##  $ weather          : chr  "sprinkles, light wind" "sprinkles, light wind" "storm entering around 2105, but doesn't start until after observations" "periods of showers" ...
##  $ fledge_date      : chr  "" "" "" "" ...
##  $ fledge_accuracy  : chr  "w/in 1 day" "w/in 1 day" "Predated" "abandoned" ...
\end{verbatim}

\begin{Shaded}
\begin{Highlighting}[]
\FunctionTok{unique}\NormalTok{(pdMF}\SpecialCharTok{$}\NormalTok{t180) }\CommentTok{\#at least one cell has an asterisk after the value}
\end{Highlighting}
\end{Shaded}

\begin{verbatim}
##  [1] "1"  "0"  ""   "4"  "4*" "2"  "9"  "10" "3"  "5"
\end{verbatim}

\begin{Shaded}
\begin{Highlighting}[]
\FunctionTok{unique}\NormalTok{(pdMF}\SpecialCharTok{$}\NormalTok{t225) }\CommentTok{\#same here}
\end{Highlighting}
\end{Shaded}

\begin{verbatim}
## [1] "1"  "0"  ""   "6*" "3"  "2"  "8"  "11" "7"
\end{verbatim}

\begin{Shaded}
\begin{Highlighting}[]
\FunctionTok{unique}\NormalTok{(pdMF}\SpecialCharTok{$}\NormalTok{nest\_age) }\CommentTok{\#"pred" and "" can be converted to NA}
\end{Highlighting}
\end{Shaded}

\begin{verbatim}
##  [1] "1"    "2"    "3"    "4"    "5"    "6"    "7"    "8"    "9"    "10"   "11"   "12"   "13"   "14"   "15"   "16"   "17"  
## [18] "18"   "19"   "20"   "21"   "22"   "23"   "24"   "25"   "26"   "27"   "28"   "29"   "30"   "31"   "32"   "33"   "34"  
## [35] "35"   "36"   "37"   "38"   "39"   "40"   "41"   "42"   "43"   "44"   "45"   "46"   "47"   "48"   "49"   "50"   "51"  
## [52] "53"   "100"  "0"    "pred" ""
\end{verbatim}

Fix structure by removing asterisks and converting to integer class. NAs
will be generated by coercion, eliminating any blank cells ("``) and
cells containing''pred" (indicating the nest was predated before the
nest age could be confirmed).

\begin{Shaded}
\begin{Highlighting}[]
\CommentTok{\#remove asterisks}
\NormalTok{pdClean }\OtherTok{\textless{}{-}}\NormalTok{ pdMF }\SpecialCharTok{\%\textgreater{}\%}
  \FunctionTok{mutate}\NormalTok{(}\AttributeTok{t180 =} \FunctionTok{gsub}\NormalTok{(}\StringTok{"}\SpecialCharTok{\textbackslash{}\textbackslash{}}\StringTok{*"}\NormalTok{, }\StringTok{""}\NormalTok{, t180)) }\SpecialCharTok{\%\textgreater{}\%}
  \FunctionTok{mutate}\NormalTok{(}\AttributeTok{t225 =} \FunctionTok{gsub}\NormalTok{(}\StringTok{"}\SpecialCharTok{\textbackslash{}\textbackslash{}}\StringTok{*"}\NormalTok{, }\StringTok{""}\NormalTok{, t225)) }


\CommentTok{\#change these columns to numeric}
\NormalTok{pdClean}\SpecialCharTok{$}\NormalTok{t180 }\OtherTok{\textless{}{-}} \FunctionTok{as.integer}\NormalTok{(pdClean}\SpecialCharTok{$}\NormalTok{t180)}
\NormalTok{pdClean}\SpecialCharTok{$}\NormalTok{t225 }\OtherTok{\textless{}{-}} \FunctionTok{as.integer}\NormalTok{(pdClean}\SpecialCharTok{$}\NormalTok{t225)}
\NormalTok{pdClean}\SpecialCharTok{$}\NormalTok{nest\_age }\OtherTok{\textless{}{-}} \FunctionTok{as.integer}\NormalTok{(pdClean}\SpecialCharTok{$}\NormalTok{nest\_age) }
\end{Highlighting}
\end{Shaded}

\begin{verbatim}
## Warning: NAs introduced by coercion
\end{verbatim}

\hypertarget{organize-data}{%
\subsection{Organize Data}\label{organize-data}}

Data was separated by sex (M vs.~F) and by incubation vs.~nestling
stage. Nestling period is defined as nest\_age \textgreater= 22 days. If
nest age was not indicated in original dataset, field notes were used to
determine whether nest was in incubation (all eggs) or nestling (at
least one nestling) stage. For these records, the following values were
manually input: nest\_age = 0 for incubation or nest\_age = 100 for
nestling, so that this data could be easily separated from known nest
age. If it was later determined that nest had been predated before
observation, ``pred'' was entered. If the nest stage could not be
determined, it was left blank.``pred'' and "" values were converted to
NA earlier when this column was converted to numeric.

\begin{Shaded}
\begin{Highlighting}[]
\CommentTok{\# Create independent dfs for M (nestling and incubation stage) and F (nestling and incubation state). }

\NormalTok{pdStage\_sex }\OtherTok{\textless{}{-}}\NormalTok{ pdClean }\SpecialCharTok{\%\textgreater{}\%}
\NormalTok{  dplyr}\SpecialCharTok{::}\FunctionTok{select}\NormalTok{(sex, nest\_age, t15}\SpecialCharTok{:}\NormalTok{t240) }\SpecialCharTok{\%\textgreater{}\%} \CommentTok{\#select relevant columns}
  \FunctionTok{mutate}\NormalTok{(}
    \AttributeTok{stage =}
      \FunctionTok{ifelse}\NormalTok{(nest\_age }\SpecialCharTok{\textless{}} \DecValTok{22}\NormalTok{, }\StringTok{"incubation"}\NormalTok{, }\StringTok{"nestling"}\NormalTok{)) }\SpecialCharTok{\%\textgreater{}\%} \CommentTok{\#add column for \textquotesingle{}stage\textquotesingle{}}
  \FunctionTok{drop\_na}\NormalTok{(stage) }\CommentTok{\#get rid of any rows blank values here, as they can\textquotesingle{}t be used for analysis}
  
\CommentTok{\#change column names to remove "t" in front of time interval}
\FunctionTok{colnames}\NormalTok{(pdStage\_sex) }\OtherTok{\textless{}{-}} \FunctionTok{c}\NormalTok{(}\StringTok{"sex"}\NormalTok{, }\StringTok{"nest\_age"}\NormalTok{, }\StringTok{"15"}\NormalTok{, }\StringTok{"30"}\NormalTok{, }\StringTok{"45"}\NormalTok{, }\StringTok{"60"}\NormalTok{, }\StringTok{"75"}\NormalTok{, }\StringTok{"90"}\NormalTok{, }\StringTok{"105"}\NormalTok{, }\StringTok{"120"}\NormalTok{, }\StringTok{"135"}\NormalTok{, }\StringTok{"150"}\NormalTok{, }\StringTok{"165"}\NormalTok{, }\StringTok{"180"}\NormalTok{, }\StringTok{"195"}\NormalTok{, }\StringTok{"210"}\NormalTok{, }\StringTok{"225"}\NormalTok{, }\StringTok{"240"}\NormalTok{, }\StringTok{"stage"}\NormalTok{)}

\CommentTok{\#create independent dfs for each study site and stage}
\NormalTok{pdM\_inc }\OtherTok{\textless{}{-}}\NormalTok{ pdStage\_sex }\SpecialCharTok{\%\textgreater{}\%}
  \FunctionTok{filter}\NormalTok{(sex }\SpecialCharTok{==}\StringTok{"M"}\NormalTok{, stage }\SpecialCharTok{==} \StringTok{"incubation"}\NormalTok{) }\SpecialCharTok{\%\textgreater{}\%}
\NormalTok{  dplyr}\SpecialCharTok{::}\FunctionTok{select}\NormalTok{(}\StringTok{\textquotesingle{}15\textquotesingle{}}\SpecialCharTok{:}\StringTok{\textquotesingle{}240\textquotesingle{}}\NormalTok{)}

\NormalTok{pdM\_nest }\OtherTok{\textless{}{-}}\NormalTok{ pdStage\_sex }\SpecialCharTok{\%\textgreater{}\%} 
  \FunctionTok{filter}\NormalTok{(sex }\SpecialCharTok{==}\StringTok{"M"}\NormalTok{, stage }\SpecialCharTok{==} \StringTok{"nestling"}\NormalTok{) }\SpecialCharTok{\%\textgreater{}\%}
\NormalTok{  dplyr}\SpecialCharTok{::}\FunctionTok{select}\NormalTok{(}\StringTok{\textquotesingle{}15\textquotesingle{}}\SpecialCharTok{:}\StringTok{\textquotesingle{}240\textquotesingle{}}\NormalTok{)}

\NormalTok{pdF\_inc }\OtherTok{\textless{}{-}}\NormalTok{ pdStage\_sex }\SpecialCharTok{\%\textgreater{}\%}
  \FunctionTok{filter}\NormalTok{(sex }\SpecialCharTok{==}\StringTok{"F"}\NormalTok{, stage }\SpecialCharTok{==} \StringTok{"incubation"}\NormalTok{) }\SpecialCharTok{\%\textgreater{}\%}
\NormalTok{  dplyr}\SpecialCharTok{::}\FunctionTok{select}\NormalTok{(}\StringTok{\textquotesingle{}15\textquotesingle{}}\SpecialCharTok{:}\StringTok{\textquotesingle{}240\textquotesingle{}}\NormalTok{)}
         
\NormalTok{pdF\_nest }\OtherTok{\textless{}{-}}\NormalTok{ pdStage\_sex }\SpecialCharTok{\%\textgreater{}\%}
  \FunctionTok{filter}\NormalTok{(sex }\SpecialCharTok{==}\StringTok{"F"}\NormalTok{, stage }\SpecialCharTok{==} \StringTok{"nestling"}\NormalTok{) }\SpecialCharTok{\%\textgreater{}\%}
\NormalTok{  dplyr}\SpecialCharTok{::}\FunctionTok{select}\NormalTok{(}\StringTok{\textquotesingle{}15\textquotesingle{}}\SpecialCharTok{:}\StringTok{\textquotesingle{}240\textquotesingle{}}\NormalTok{)}
\end{Highlighting}
\end{Shaded}

\hypertarget{mean-pd-tables}{%
\subsection{Mean PD tables}\label{mean-pd-tables}}

Create four stand-alone data.frames, one for M (incubation), one for M
(nestling), one for F (incubation), and one for F (nestling). These will
be used for t-tests.

\begin{Shaded}
\begin{Highlighting}[]
\NormalTok{meanM\_inc }\OtherTok{\textless{}{-}} \FunctionTok{data.frame}\NormalTok{(}
  \AttributeTok{time =} \FunctionTok{as.numeric}\NormalTok{(}\FunctionTok{colnames}\NormalTok{(pdM\_inc)),}
  \AttributeTok{M\_incubation =} \FunctionTok{colMeans}\NormalTok{(pdM\_inc, }\AttributeTok{na.rm =} \ConstantTok{TRUE}\NormalTok{))}

\NormalTok{meanM\_nest }\OtherTok{\textless{}{-}} \FunctionTok{data.frame}\NormalTok{(}
  \AttributeTok{time =} \FunctionTok{as.numeric}\NormalTok{(}\FunctionTok{colnames}\NormalTok{(pdM\_nest)),}
  \AttributeTok{M\_nestling =} \FunctionTok{colMeans}\NormalTok{(pdM\_nest, }\AttributeTok{na.rm =} \ConstantTok{TRUE}\NormalTok{))}

\NormalTok{meanF\_inc }\OtherTok{\textless{}{-}} \FunctionTok{data.frame}\NormalTok{(}
  \AttributeTok{time =} \FunctionTok{as.numeric}\NormalTok{(}\FunctionTok{colnames}\NormalTok{(pdF\_inc)),}
  \AttributeTok{F\_incubation =} \FunctionTok{colMeans}\NormalTok{(pdF\_inc, }\AttributeTok{na.rm =} \ConstantTok{TRUE}\NormalTok{))}

\NormalTok{meanF\_nest }\OtherTok{\textless{}{-}} \FunctionTok{data.frame}\NormalTok{(}
  \AttributeTok{time =} \FunctionTok{as.numeric}\NormalTok{(}\FunctionTok{colnames}\NormalTok{(pdF\_nest)),}
  \AttributeTok{F\_nestling =} \FunctionTok{colMeans}\NormalTok{(pdF\_nest, }\AttributeTok{na.rm =} \ConstantTok{TRUE}\NormalTok{))}
\end{Highlighting}
\end{Shaded}

\hypertarget{calculate-confidence-intervals}{%
\subsection{Calculate confidence
intervals}\label{calculate-confidence-intervals}}

Apply getCI function across data.frames for each sex and stage.

\begin{Shaded}
\begin{Highlighting}[]
\NormalTok{ciM\_inc }\OtherTok{\textless{}{-}} \FunctionTok{apply}\NormalTok{(pdM\_inc, }\DecValTok{2}\NormalTok{, }\AttributeTok{FUN =}\NormalTok{ getCI)}
\NormalTok{ciM\_nest }\OtherTok{\textless{}{-}} \FunctionTok{apply}\NormalTok{(pdM\_nest, }\DecValTok{2}\NormalTok{, }\AttributeTok{FUN =}\NormalTok{ getCI)}
\NormalTok{ciF\_inc }\OtherTok{\textless{}{-}} \FunctionTok{apply}\NormalTok{(pdF\_inc, }\DecValTok{2}\NormalTok{, }\AttributeTok{FUN =}\NormalTok{ getCI)}
\NormalTok{ciF\_nest }\OtherTok{\textless{}{-}} \FunctionTok{apply}\NormalTok{(pdF\_nest, }\DecValTok{2}\NormalTok{, }\AttributeTok{FUN =}\NormalTok{ getCI)}
\end{Highlighting}
\end{Shaded}

Create new data.frames (one for nestling stage and one for incubation
stage) with means and CIs. Remove rows where time \textgreater{} 180 for
ciInc because no data is available for MGSA after this time.

\begin{Shaded}
\begin{Highlighting}[]
\NormalTok{ciInc\_sex }\OtherTok{\textless{}{-}} \FunctionTok{data.frame}\NormalTok{(}
    \AttributeTok{sex =} \FunctionTok{c}\NormalTok{(}\FunctionTok{rep}\NormalTok{(}\StringTok{"M"}\NormalTok{, }\FunctionTok{nrow}\NormalTok{(meanM\_inc)), }\FunctionTok{rep}\NormalTok{(}\StringTok{"F"}\NormalTok{, }\FunctionTok{nrow}\NormalTok{(meanF\_inc))),}
    \AttributeTok{mean =} \FunctionTok{c}\NormalTok{(}\FunctionTok{colMeans}\NormalTok{(pdM\_inc, }\AttributeTok{na.rm =} \ConstantTok{TRUE}\NormalTok{), }\FunctionTok{colMeans}\NormalTok{(pdF\_inc, }\AttributeTok{na.rm =} \ConstantTok{TRUE}\NormalTok{)),}
    \AttributeTok{ci\_l =} \FunctionTok{c}\NormalTok{(ciM\_inc[}\DecValTok{1}\NormalTok{,], ciF\_inc[}\DecValTok{1}\NormalTok{,]),}
    \AttributeTok{ci\_h =} \FunctionTok{c}\NormalTok{(ciM\_inc[}\DecValTok{2}\NormalTok{,], ciF\_inc[}\DecValTok{2}\NormalTok{,]),}
    \AttributeTok{time =} \FunctionTok{c}\NormalTok{(}\FunctionTok{as.numeric}\NormalTok{(}\FunctionTok{rownames}\NormalTok{(meanM\_inc)), }\FunctionTok{as.numeric}\NormalTok{(}\FunctionTok{rownames}\NormalTok{(meanF\_inc))),}
    \AttributeTok{stage =} \StringTok{"Incubation"}\NormalTok{)}


\NormalTok{ciNest\_sex }\OtherTok{\textless{}{-}} \FunctionTok{data.frame}\NormalTok{(}
    \AttributeTok{sex =} \FunctionTok{c}\NormalTok{(}\FunctionTok{rep}\NormalTok{(}\StringTok{"M"}\NormalTok{, }\FunctionTok{nrow}\NormalTok{(meanM\_nest)), }\FunctionTok{rep}\NormalTok{(}\StringTok{"F"}\NormalTok{, }\FunctionTok{nrow}\NormalTok{(meanF\_nest))),}
    \AttributeTok{mean =} \FunctionTok{c}\NormalTok{(}\FunctionTok{colMeans}\NormalTok{(pdM\_nest, }\AttributeTok{na.rm =} \ConstantTok{TRUE}\NormalTok{), }\FunctionTok{colMeans}\NormalTok{(pdF\_nest, }\AttributeTok{na.rm =} \ConstantTok{TRUE}\NormalTok{)),}
    \AttributeTok{ci\_l =} \FunctionTok{c}\NormalTok{(ciM\_nest[}\DecValTok{1}\NormalTok{,], ciF\_nest[}\DecValTok{1}\NormalTok{,]),}
    \AttributeTok{ci\_h =} \FunctionTok{c}\NormalTok{(ciM\_nest[}\DecValTok{2}\NormalTok{,], ciF\_nest[}\DecValTok{2}\NormalTok{,]),}
    \AttributeTok{time =} \FunctionTok{c}\NormalTok{(}\FunctionTok{as.numeric}\NormalTok{(}\FunctionTok{rownames}\NormalTok{(meanM\_nest)), }\FunctionTok{as.numeric}\NormalTok{(}\FunctionTok{rownames}\NormalTok{(meanF\_nest))),}
    \AttributeTok{stage =} \StringTok{"Nestling"}\NormalTok{)}
    
\NormalTok{ciAll\_sex }\OtherTok{\textless{}{-}} \FunctionTok{rbind}\NormalTok{(ciInc\_sex, ciNest\_sex)}
\end{Highlighting}
\end{Shaded}

\hypertarget{test-for-difference}{%
\subsection{Test for Difference}\label{test-for-difference}}

Test for normality for each time interval in M\_inc, M\_nest, and
F\_nest (excluded pdF\_inc because all values = 0, as female did not
deliver prey while incubating).

\begin{Shaded}
\begin{Highlighting}[]
\CommentTok{\# Store results of testNormality in new df}
\NormalTok{normality\_sex }\OtherTok{\textless{}{-}} \FunctionTok{data.frame}\NormalTok{(}
  \AttributeTok{M\_inc =} \FunctionTok{testNormality}\NormalTok{(pdM\_inc, }\DecValTok{2}\NormalTok{), }
  \AttributeTok{M\_nest =} \FunctionTok{testNormality}\NormalTok{(pdM\_nest, }\DecValTok{2}\NormalTok{), }
  \AttributeTok{F\_nest =} \FunctionTok{testNormality}\NormalTok{(pdF\_nest, }\DecValTok{2}\NormalTok{)}
\NormalTok{  )}

\CommentTok{\# Test if any values are not significant}
\FunctionTok{any}\NormalTok{(normality\_sex }\SpecialCharTok{\textgreater{}=} \FloatTok{0.05}\NormalTok{)}
\end{Highlighting}
\end{Shaded}

\begin{verbatim}
## [1] FALSE
\end{verbatim}

All vectors in each list are significantly different from null
hypothesis of Shapiro-Wilk normality test, meaning that no vector is
normally distributed. Therefore, we use the unpaired two-sample Wilcoxon
test to compare median values of M and F prey deliveries during each
time interval, for incubation and nestling stages.

\begin{Shaded}
\begin{Highlighting}[]
\CommentTok{\# Incubation Stage}
\NormalTok{wilcox\_inc\_sex }\OtherTok{\textless{}{-}} \FunctionTok{testWilcox}\NormalTok{(pdM\_inc, pdF\_inc)}

\FunctionTok{any}\NormalTok{(wilcox\_inc\_sex}\SpecialCharTok{$}\NormalTok{values }\SpecialCharTok{\textless{}=} \FloatTok{0.05}\NormalTok{) }\CommentTok{\#test if any p.values are significantly different: TRUE}
\end{Highlighting}
\end{Shaded}

\begin{verbatim}
## [1] TRUE
\end{verbatim}

\begin{Shaded}
\begin{Highlighting}[]
\FunctionTok{all}\NormalTok{(wilcox\_inc\_sex}\SpecialCharTok{$}\NormalTok{values }\SpecialCharTok{\textless{}=} \FloatTok{0.05}\NormalTok{) }\CommentTok{\#test if all p.values are significantly different: FALSE}
\end{Highlighting}
\end{Shaded}

\begin{verbatim}
## [1] FALSE
\end{verbatim}

\begin{Shaded}
\begin{Highlighting}[]
\NormalTok{dplyr}\SpecialCharTok{::}\FunctionTok{filter}\NormalTok{(wilcox\_inc\_sex, values }\SpecialCharTok{\textless{}=} \FloatTok{0.05}\NormalTok{) }\CommentTok{\#print which rows have p.value that is significantly different}
\end{Highlighting}
\end{Shaded}

\begin{verbatim}
##          values ind
## 1  2.054022e-06  15
## 2  2.962959e-08  30
## 3  1.662023e-14  45
## 4  1.139046e-19  60
## 5  6.363476e-15  75
## 6  4.309115e-13  90
## 7  1.333988e-08 105
## 8  1.102312e-06 120
## 9  3.546280e-05 135
## 10 1.193741e-05 150
## 11 3.625618e-03 165
## 12 9.670479e-04 180
## 13 3.590980e-02 195
## 14 8.858782e-03 210
\end{verbatim}

\begin{Shaded}
\begin{Highlighting}[]
\CommentTok{\# Nestling Stage}
\NormalTok{wilcox\_nest\_sex }\OtherTok{\textless{}{-}} \FunctionTok{testWilcox}\NormalTok{(pdM\_nest, pdF\_nest)}

\FunctionTok{any}\NormalTok{(wilcox\_nest\_sex}\SpecialCharTok{$}\NormalTok{values }\SpecialCharTok{\textless{}=} \FloatTok{0.05}\NormalTok{) }\CommentTok{\#test if any p.values are significantly different: TRUE}
\end{Highlighting}
\end{Shaded}

\begin{verbatim}
## [1] TRUE
\end{verbatim}

\begin{Shaded}
\begin{Highlighting}[]
\FunctionTok{all}\NormalTok{(wilcox\_nest\_sex}\SpecialCharTok{$}\NormalTok{values }\SpecialCharTok{\textless{}=} \FloatTok{0.05}\NormalTok{) }\CommentTok{\#test if all p.values are significantly different: TRUE}
\end{Highlighting}
\end{Shaded}

\begin{verbatim}
## [1] TRUE
\end{verbatim}

From the Wilcoxon tests, we see that all median time intervals during
the nestling stage are significantly different between males and
females. However, during the incubation stage, median prey delivery
rates at time = 2255 and time = 240 are not significantly different
between males and females.

\hypertarget{plot}{%
\subsection{Plot}\label{plot}}

Incubation (including star indicating non-significant Wilcoxon test at t
= 225 and t = 240):

\begin{Shaded}
\begin{Highlighting}[]
\NormalTok{plotInc\_sex }\OtherTok{\textless{}{-}} \FunctionTok{ggplot}\NormalTok{(}\AttributeTok{data =}\NormalTok{ ciInc\_sex) }\SpecialCharTok{+}
  \FunctionTok{geom\_point}\NormalTok{(}\FunctionTok{aes}\NormalTok{(}\AttributeTok{x =}\NormalTok{ time, }\AttributeTok{y =}\NormalTok{ mean, }\AttributeTok{color =}\NormalTok{ sex, }\AttributeTok{group =}\NormalTok{ sex),}
             \AttributeTok{position =} \FunctionTok{position\_dodge}\NormalTok{(}\AttributeTok{width=}\FloatTok{0.75}\NormalTok{)) }\SpecialCharTok{+}
  \FunctionTok{geom\_errorbar}\NormalTok{(}\FunctionTok{aes}\NormalTok{(}\AttributeTok{x=}\NormalTok{time, }\AttributeTok{ymax =}\NormalTok{ ci\_h, }\AttributeTok{ymin=}\NormalTok{ci\_l, }\AttributeTok{color =}\NormalTok{ sex, }
                    \AttributeTok{group =}\NormalTok{ sex),}
                \AttributeTok{position =} \FunctionTok{position\_dodge}\NormalTok{(}\AttributeTok{width=}\FloatTok{0.75}\NormalTok{)) }\SpecialCharTok{+}
  \FunctionTok{labs}\NormalTok{(}\AttributeTok{x =} \StringTok{"Time After Sunset (minutes)"}\NormalTok{, }\AttributeTok{y =} \StringTok{"Mean Prey Deliveries"}\NormalTok{, }
       \AttributeTok{title =} \StringTok{"Incubation"}\NormalTok{, }\AttributeTok{color =} \StringTok{"Sex"}\NormalTok{) }\SpecialCharTok{+}
  \FunctionTok{theme\_minimal}\NormalTok{() }\SpecialCharTok{+}
  \FunctionTok{theme}\NormalTok{(}\AttributeTok{plot.title =} \FunctionTok{element\_text}\NormalTok{(}\AttributeTok{hjust =} \FloatTok{0.5}\NormalTok{)) }\SpecialCharTok{+}
  \FunctionTok{geom\_point}\NormalTok{(}\FunctionTok{aes}\NormalTok{(}\AttributeTok{x =} \DecValTok{225}\NormalTok{, }\AttributeTok{y =} \FloatTok{4.5}\NormalTok{), }\AttributeTok{shape =} \DecValTok{8}\NormalTok{, }\AttributeTok{stroke =} \FloatTok{0.1}\NormalTok{) }\SpecialCharTok{+}
  \FunctionTok{geom\_point}\NormalTok{(}\FunctionTok{aes}\NormalTok{(}\AttributeTok{x =} \DecValTok{240}\NormalTok{, }\AttributeTok{y =} \FloatTok{3.5}\NormalTok{), }\AttributeTok{shape =} \DecValTok{8}\NormalTok{, }\AttributeTok{stroke =} \FloatTok{0.1}\NormalTok{)}
\end{Highlighting}
\end{Shaded}

Nestling:

\begin{Shaded}
\begin{Highlighting}[]
\NormalTok{plotNest\_sex }\OtherTok{\textless{}{-}} \FunctionTok{ggplot}\NormalTok{(}\AttributeTok{data =}\NormalTok{ ciNest\_sex) }\SpecialCharTok{+}
  \FunctionTok{geom\_point}\NormalTok{(}\FunctionTok{aes}\NormalTok{(}\AttributeTok{x =}\NormalTok{ time, }\AttributeTok{y =}\NormalTok{ mean, }\AttributeTok{color =}\NormalTok{ sex, }\AttributeTok{group =}\NormalTok{ sex),}
             \AttributeTok{position =} \FunctionTok{position\_dodge}\NormalTok{(}\AttributeTok{width=}\FloatTok{0.75}\NormalTok{)) }\SpecialCharTok{+}
  \FunctionTok{geom\_errorbar}\NormalTok{(}\FunctionTok{aes}\NormalTok{(}\AttributeTok{x=}\NormalTok{time, }\AttributeTok{ymax =}\NormalTok{ ci\_h, }\AttributeTok{ymin=}\NormalTok{ci\_l, }\AttributeTok{color =}\NormalTok{ sex, }
                    \AttributeTok{group =}\NormalTok{ sex),}
                \AttributeTok{position =} \FunctionTok{position\_dodge}\NormalTok{(}\AttributeTok{width=}\FloatTok{0.75}\NormalTok{)) }\SpecialCharTok{+}
  \FunctionTok{labs}\NormalTok{(}\AttributeTok{x =} \StringTok{"Time After Sunset (minutes)"}\NormalTok{, }\AttributeTok{y =} \StringTok{"Mean Prey Deliveries"}\NormalTok{, }
       \AttributeTok{title =} \StringTok{"Nestling"}\NormalTok{, }\AttributeTok{color =} \StringTok{"Sex"}\NormalTok{) }\SpecialCharTok{+}
  \FunctionTok{theme\_minimal}\NormalTok{() }\SpecialCharTok{+}
  \FunctionTok{theme}\NormalTok{(}\AttributeTok{plot.title =} \FunctionTok{element\_text}\NormalTok{(}\AttributeTok{hjust =} \FloatTok{0.5}\NormalTok{)) }
\end{Highlighting}
\end{Shaded}

The final plot compares both incubation and nestling stages.

\begin{Shaded}
\begin{Highlighting}[]
\FunctionTok{grid.arrange}\NormalTok{(plotInc\_sex, plotNest\_sex)}
\end{Highlighting}
\end{Shaded}

\begin{figure}
\centering
\includegraphics{pd_analysis_files/figure-latex/final_plot_sex-1.pdf}
\caption{Mean prey deliveries during the incubation stage (top) and
nestling stage (bottom). Males are shown in bue and females in red.
Stars indicate non-significant differences in medians based on unpaired
two-sample Wilcoxon tests.}
\end{figure}

\hypertarget{prey-delivery-rates-by-site}{%
\section{Prey Delivery Rates by
Site}\label{prey-delivery-rates-by-site}}

The original spreadsheet contains prey delivery data from two other
studies that were not considered in this analysis. Therefore, they were
filtered out, and ``B'' (MGSA) and ``C'' (HFSA) were preserved for the
analysis. Females were also excluded because here was not sufficient
data for reliable analysis.

\begin{Shaded}
\begin{Highlighting}[]
\NormalTok{pdHM }\OtherTok{\textless{}{-}}\NormalTok{ pdClean }\SpecialCharTok{\%\textgreater{}\%}
  \FunctionTok{filter}\NormalTok{(study\_site }\SpecialCharTok{==} \StringTok{"B"} \SpecialCharTok{|}\NormalTok{ study\_site }\SpecialCharTok{==} \StringTok{"C"}\NormalTok{, sex }\SpecialCharTok{==} \StringTok{"M"}\NormalTok{)}
\end{Highlighting}
\end{Shaded}

\hypertarget{organize-data-1}{%
\subsection{Organize Data}\label{organize-data-1}}

Data was separated by study site (HFSA vs.~MGSA) and by incubation
vs.~nestling stage.

since the last observation for HFSA is at time = 240, we'll end both
datasets there.

First, create data frame with PDs for whole dataset, then broken down by
each study site and stage. End incubation data frames at t = 180 and
nesting data frames at t = 240 because this is the maximum time when
there is data for both sites.

\begin{Shaded}
\begin{Highlighting}[]
\CommentTok{\#create independet dfs for HFSA (nestling and incubation stage) and MGSA (nestling and incubation state). }

\CommentTok{\#select relevant columns, add column for stage, rename study sites, drop NAs}
\NormalTok{pdStage\_site }\OtherTok{\textless{}{-}}\NormalTok{ pdClean }\SpecialCharTok{\%\textgreater{}\%}
\NormalTok{  dplyr}\SpecialCharTok{::}\FunctionTok{select}\NormalTok{(study\_site, nest\_age, t15}\SpecialCharTok{:}\NormalTok{t240) }\SpecialCharTok{\%\textgreater{}\%}
  \FunctionTok{mutate}\NormalTok{(}
    \AttributeTok{stage =}
      \FunctionTok{ifelse}\NormalTok{(nest\_age }\SpecialCharTok{\textless{}} \DecValTok{22}\NormalTok{, }\StringTok{"incubation"}\NormalTok{, }\StringTok{"nestling"}\NormalTok{),}
    \AttributeTok{study\_site =}
      \FunctionTok{ifelse}\NormalTok{(study\_site }\SpecialCharTok{==} \StringTok{"B"}\NormalTok{, }\StringTok{"MGSA"}\NormalTok{, }\StringTok{"HFSA"}\NormalTok{)) }\SpecialCharTok{\%\textgreater{}\%}
  \FunctionTok{drop\_na}\NormalTok{(stage)}

\CommentTok{\#change column names to remove "t" in front of time interval}
\FunctionTok{colnames}\NormalTok{(pdStage\_site) }\OtherTok{\textless{}{-}} \FunctionTok{c}\NormalTok{(}\StringTok{"study\_site"}\NormalTok{, }\StringTok{"nest\_age"}\NormalTok{, }\StringTok{"15"}\NormalTok{, }\StringTok{"30"}\NormalTok{, }\StringTok{"45"}\NormalTok{, }\StringTok{"60"}\NormalTok{, }\StringTok{"75"}\NormalTok{, }\StringTok{"90"}\NormalTok{, }\StringTok{"105"}\NormalTok{, }\StringTok{"120"}\NormalTok{, }\StringTok{"135"}\NormalTok{, }\StringTok{"150"}\NormalTok{, }\StringTok{"165"}\NormalTok{, }\StringTok{"180"}\NormalTok{, }\StringTok{"195"}\NormalTok{, }\StringTok{"210"}\NormalTok{, }\StringTok{"225"}\NormalTok{, }\StringTok{"240"}\NormalTok{, }\StringTok{"stage"}\NormalTok{)}

\CommentTok{\#create independent dfs for each study site and stage}
\NormalTok{pdHFSA\_inc }\OtherTok{\textless{}{-}}\NormalTok{ pdStage\_site }\SpecialCharTok{\%\textgreater{}\%}
  \FunctionTok{filter}\NormalTok{(study\_site }\SpecialCharTok{==}\StringTok{"HFSA"}\NormalTok{, stage }\SpecialCharTok{==} \StringTok{"incubation"}\NormalTok{) }\SpecialCharTok{\%\textgreater{}\%}
\NormalTok{  dplyr}\SpecialCharTok{::}\FunctionTok{select}\NormalTok{(}\StringTok{\textquotesingle{}15\textquotesingle{}}\SpecialCharTok{:}\StringTok{\textquotesingle{}180\textquotesingle{}}\NormalTok{)}

\NormalTok{pdHFSA\_nest }\OtherTok{\textless{}{-}}\NormalTok{ pdStage\_site }\SpecialCharTok{\%\textgreater{}\%} 
  \FunctionTok{filter}\NormalTok{(study\_site }\SpecialCharTok{==}\StringTok{"HFSA"}\NormalTok{, stage }\SpecialCharTok{==} \StringTok{"nestling"}\NormalTok{) }\SpecialCharTok{\%\textgreater{}\%}
\NormalTok{  dplyr}\SpecialCharTok{::}\FunctionTok{select}\NormalTok{(}\StringTok{\textquotesingle{}15\textquotesingle{}}\SpecialCharTok{:}\StringTok{\textquotesingle{}240\textquotesingle{}}\NormalTok{)}

\NormalTok{pdMGSA\_inc }\OtherTok{\textless{}{-}}\NormalTok{ pdStage\_site }\SpecialCharTok{\%\textgreater{}\%}
  \FunctionTok{filter}\NormalTok{(study\_site }\SpecialCharTok{==}\StringTok{"MGSA"}\NormalTok{, stage }\SpecialCharTok{==} \StringTok{"incubation"}\NormalTok{) }\SpecialCharTok{\%\textgreater{}\%}
\NormalTok{  dplyr}\SpecialCharTok{::}\FunctionTok{select}\NormalTok{(}\StringTok{\textquotesingle{}15\textquotesingle{}}\SpecialCharTok{:}\StringTok{\textquotesingle{}180\textquotesingle{}}\NormalTok{)}
         
\NormalTok{pdMGSA\_nest }\OtherTok{\textless{}{-}}\NormalTok{ pdStage\_site }\SpecialCharTok{\%\textgreater{}\%}
  \FunctionTok{filter}\NormalTok{(study\_site }\SpecialCharTok{==}\StringTok{"MGSA"}\NormalTok{, stage }\SpecialCharTok{==} \StringTok{"nestling"}\NormalTok{) }\SpecialCharTok{\%\textgreater{}\%}
\NormalTok{  dplyr}\SpecialCharTok{::}\FunctionTok{select}\NormalTok{(}\StringTok{\textquotesingle{}15\textquotesingle{}}\SpecialCharTok{:}\StringTok{\textquotesingle{}240\textquotesingle{}}\NormalTok{)}
\end{Highlighting}
\end{Shaded}

\hypertarget{mean-pd-tables-1}{%
\subsection{Mean PD tables}\label{mean-pd-tables-1}}

Create four stand-alone data.frames, one for HFSA (incubation), one for
HFSA (nestling), one for MGSA (incubation), and one for MGSA (nestling).

\begin{Shaded}
\begin{Highlighting}[]
\NormalTok{meanHFSA\_inc }\OtherTok{\textless{}{-}} \FunctionTok{data.frame}\NormalTok{(}
  \AttributeTok{time =} \FunctionTok{as.numeric}\NormalTok{(}\FunctionTok{colnames}\NormalTok{(pdHFSA\_inc)),}
  \AttributeTok{HFSA\_incubation =} \FunctionTok{colMeans}\NormalTok{(pdHFSA\_inc, }\AttributeTok{na.rm =} \ConstantTok{TRUE}\NormalTok{))}

\NormalTok{meanHFSA\_nest }\OtherTok{\textless{}{-}} \FunctionTok{data.frame}\NormalTok{(}
  \AttributeTok{time =} \FunctionTok{as.numeric}\NormalTok{(}\FunctionTok{colnames}\NormalTok{(pdHFSA\_nest)),}
  \AttributeTok{HFSA\_nestling =} \FunctionTok{colMeans}\NormalTok{(pdHFSA\_nest, }\AttributeTok{na.rm =} \ConstantTok{TRUE}\NormalTok{))}

\NormalTok{meanMGSA\_inc }\OtherTok{\textless{}{-}} \FunctionTok{data.frame}\NormalTok{(}
  \AttributeTok{time =} \FunctionTok{as.numeric}\NormalTok{(}\FunctionTok{colnames}\NormalTok{(pdMGSA\_inc)),}
  \AttributeTok{MGSA\_incubation =} \FunctionTok{colMeans}\NormalTok{(pdMGSA\_inc, }\AttributeTok{na.rm =} \ConstantTok{TRUE}\NormalTok{))}

\NormalTok{meanMGSA\_nest }\OtherTok{\textless{}{-}} \FunctionTok{data.frame}\NormalTok{(}
  \AttributeTok{time =} \FunctionTok{as.numeric}\NormalTok{(}\FunctionTok{colnames}\NormalTok{(pdMGSA\_nest)),}
  \AttributeTok{MGSA\_nestling =} \FunctionTok{colMeans}\NormalTok{(pdMGSA\_nest, }\AttributeTok{na.rm =} \ConstantTok{TRUE}\NormalTok{))}
\end{Highlighting}
\end{Shaded}

\hypertarget{calculate-confidence-intervals-1}{%
\subsection{Calculate confidence
intervals}\label{calculate-confidence-intervals-1}}

Apply getCI across each data.frame, taking each time interval as vector.

\begin{Shaded}
\begin{Highlighting}[]
\NormalTok{ciHFSA\_inc }\OtherTok{\textless{}{-}} \FunctionTok{apply}\NormalTok{(pdHFSA\_inc, }\DecValTok{2}\NormalTok{, }\AttributeTok{FUN =}\NormalTok{ getCI)}
\NormalTok{ciHFSA\_nest }\OtherTok{\textless{}{-}} \FunctionTok{apply}\NormalTok{(pdHFSA\_nest, }\DecValTok{2}\NormalTok{, }\AttributeTok{FUN =}\NormalTok{ getCI)}
\NormalTok{ciMGSA\_inc }\OtherTok{\textless{}{-}} \FunctionTok{apply}\NormalTok{(pdMGSA\_inc, }\DecValTok{2}\NormalTok{, }\AttributeTok{FUN =}\NormalTok{ getCI)}
\NormalTok{ciMGSA\_nest }\OtherTok{\textless{}{-}} \FunctionTok{apply}\NormalTok{(pdMGSA\_nest, }\DecValTok{2}\NormalTok{, }\AttributeTok{FUN =}\NormalTok{ getCI)}
\end{Highlighting}
\end{Shaded}

Add CIs to a data frame.

\begin{Shaded}
\begin{Highlighting}[]
\NormalTok{ciInc }\OtherTok{\textless{}{-}} \FunctionTok{data.frame}\NormalTok{(}
    \AttributeTok{study\_area =} \FunctionTok{c}\NormalTok{(}\FunctionTok{rep}\NormalTok{(}\StringTok{"HFSA"}\NormalTok{, }\FunctionTok{nrow}\NormalTok{(meanHFSA\_inc)), }\FunctionTok{rep}\NormalTok{(}\StringTok{"MGSA"}\NormalTok{, }\FunctionTok{nrow}\NormalTok{(meanMGSA\_inc))),}
    \AttributeTok{mean =} \FunctionTok{c}\NormalTok{(}\FunctionTok{colMeans}\NormalTok{(pdHFSA\_inc, }\AttributeTok{na.rm =} \ConstantTok{TRUE}\NormalTok{), }\FunctionTok{colMeans}\NormalTok{(pdMGSA\_inc, }\AttributeTok{na.rm =} \ConstantTok{TRUE}\NormalTok{)),}
    \AttributeTok{ci\_l =} \FunctionTok{c}\NormalTok{(ciHFSA\_inc[}\DecValTok{1}\NormalTok{,], ciMGSA\_inc[}\DecValTok{1}\NormalTok{,]),}
    \AttributeTok{ci\_h =} \FunctionTok{c}\NormalTok{(ciHFSA\_inc[}\DecValTok{2}\NormalTok{,], ciMGSA\_inc[}\DecValTok{2}\NormalTok{,]),}
    \AttributeTok{time =} \FunctionTok{c}\NormalTok{(}\FunctionTok{as.numeric}\NormalTok{(}\FunctionTok{rownames}\NormalTok{(meanHFSA\_inc)), }\FunctionTok{as.numeric}\NormalTok{(}\FunctionTok{rownames}\NormalTok{(meanMGSA\_inc))),}
    \AttributeTok{stage =} \StringTok{"Incubation"}\NormalTok{)}


\NormalTok{ciNest }\OtherTok{\textless{}{-}} \FunctionTok{data.frame}\NormalTok{(}
    \AttributeTok{study\_area =} \FunctionTok{c}\NormalTok{(}\FunctionTok{rep}\NormalTok{(}\StringTok{"HFSA"}\NormalTok{, }\FunctionTok{nrow}\NormalTok{(meanHFSA\_nest)), }\FunctionTok{rep}\NormalTok{(}\StringTok{"MGSA"}\NormalTok{, }\FunctionTok{nrow}\NormalTok{(meanMGSA\_nest))),}
    \AttributeTok{mean =} \FunctionTok{c}\NormalTok{(}\FunctionTok{colMeans}\NormalTok{(pdHFSA\_nest, }\AttributeTok{na.rm =} \ConstantTok{TRUE}\NormalTok{), }\FunctionTok{colMeans}\NormalTok{(pdMGSA\_nest, }\AttributeTok{na.rm =} \ConstantTok{TRUE}\NormalTok{)),}
    \AttributeTok{ci\_l =} \FunctionTok{c}\NormalTok{(ciHFSA\_nest[}\DecValTok{1}\NormalTok{,], ciMGSA\_nest[}\DecValTok{1}\NormalTok{,]),}
    \AttributeTok{ci\_h =} \FunctionTok{c}\NormalTok{(ciHFSA\_nest[}\DecValTok{2}\NormalTok{,], ciMGSA\_nest[}\DecValTok{2}\NormalTok{,]),}
    \AttributeTok{time =} \FunctionTok{c}\NormalTok{(}\FunctionTok{as.numeric}\NormalTok{(}\FunctionTok{rownames}\NormalTok{(meanHFSA\_nest)), }\FunctionTok{as.numeric}\NormalTok{(}\FunctionTok{rownames}\NormalTok{(meanMGSA\_nest))),}
    \AttributeTok{stage =} \StringTok{"Nestling"}\NormalTok{)}
    
\NormalTok{ciAll }\OtherTok{\textless{}{-}} \FunctionTok{rbind}\NormalTok{(ciInc, ciNest)}
\end{Highlighting}
\end{Shaded}

\hypertarget{test-for-difference-1}{%
\subsection{Test for Difference}\label{test-for-difference-1}}

Test for normality for each time interval in pdMGSA\_inc, pdMGSA\_nest,
pdHFSA\_inc, and pdHFSA\_nest.

\begin{Shaded}
\begin{Highlighting}[]
\CommentTok{\# Store results of testNormality in new df}
\NormalTok{normality\_site }\OtherTok{\textless{}{-}} \FunctionTok{list}\NormalTok{(}
  \AttributeTok{MGSA\_inc =} \FunctionTok{testNormality}\NormalTok{(pdMGSA\_inc, }\DecValTok{2}\NormalTok{), }
  \AttributeTok{MGSA\_nest =} \FunctionTok{testNormality}\NormalTok{(pdMGSA\_nest, }\DecValTok{2}\NormalTok{), }
  \AttributeTok{HFSA\_inc =} \FunctionTok{testNormality}\NormalTok{(pdHFSA\_inc, }\DecValTok{2}\NormalTok{),}
  \AttributeTok{HFSA\_nest =} \FunctionTok{testNormality}\NormalTok{(pdHFSA\_nest, }\DecValTok{2}\NormalTok{))}

\CommentTok{\# Test if any values are not significant}
\FunctionTok{sapply}\NormalTok{(normality\_site, }\AttributeTok{FUN =} \ControlFlowTok{function}\NormalTok{(x)\{}\FunctionTok{any}\NormalTok{(x }\SpecialCharTok{\textgreater{}=} \FloatTok{0.05}\NormalTok{)\})}
\end{Highlighting}
\end{Shaded}

\begin{verbatim}
##  MGSA_inc MGSA_nest  HFSA_inc HFSA_nest 
##     FALSE     FALSE     FALSE     FALSE
\end{verbatim}

All vectors in each list are significantly different from null
hypothesis of Shapiro-Wilk normality test, meaning that no vector is
normally distributed. Therefore, we use the unpaired two-sample Wilcoxon
test to compare median values for each time interval in MGSA and HFSA,
for incubation and nestling stages.

\begin{Shaded}
\begin{Highlighting}[]
\CommentTok{\# Incubation Stage}
\NormalTok{wilcox\_inc }\OtherTok{\textless{}{-}} \FunctionTok{testWilcox}\NormalTok{(pdMGSA\_inc, pdHFSA\_inc)}
\FunctionTok{print}\NormalTok{(wilcox\_inc)}
\end{Highlighting}
\end{Shaded}

\begin{verbatim}
##       values ind
## 1  0.8856867  15
## 2  0.5688739  30
## 3  0.3943537  45
## 4  0.9802463  60
## 5  0.2187071  75
## 6  0.3430303  90
## 7  0.4221215 105
## 8  0.9601425 120
## 9  0.1674084 135
## 10 0.1482079 150
## 11 0.5055388 165
## 12 0.6702783 180
\end{verbatim}

\begin{Shaded}
\begin{Highlighting}[]
\FunctionTok{any}\NormalTok{(wilcox\_inc}\SpecialCharTok{$}\NormalTok{values }\SpecialCharTok{\textless{}=} \FloatTok{0.05}\NormalTok{) }\CommentTok{\#test if any p.values are significantly different: FALSE}
\end{Highlighting}
\end{Shaded}

\begin{verbatim}
## [1] FALSE
\end{verbatim}

\begin{Shaded}
\begin{Highlighting}[]
\FunctionTok{all}\NormalTok{(wilcox\_inc}\SpecialCharTok{$}\NormalTok{values }\SpecialCharTok{\textgreater{}} \FloatTok{0.05}\NormalTok{) }\CommentTok{\#test if all p.values are not significant (double{-}checking first logical expression\textquotesingle{}s validity): TRUE}
\end{Highlighting}
\end{Shaded}

\begin{verbatim}
## [1] TRUE
\end{verbatim}

\begin{Shaded}
\begin{Highlighting}[]
\CommentTok{\# Nestling Stage}
\NormalTok{wilcox\_nest }\OtherTok{\textless{}{-}} \FunctionTok{testWilcox}\NormalTok{(pdMGSA\_nest, pdHFSA\_nest)}
\FunctionTok{print}\NormalTok{(wilcox\_nest)}
\end{Highlighting}
\end{Shaded}

\begin{verbatim}
##        values ind
## 1  0.94970196  15
## 2  0.67158087  30
## 3  0.22037822  45
## 4  0.43641197  60
## 5  0.13753444  75
## 6  0.45695064  90
## 7  0.48124642 105
## 8  0.11508571 120
## 9  0.83764080 135
## 10 0.13535971 150
## 11 0.01047295 165
## 12 0.77862619 180
## 13 0.20597828 195
## 14 0.40201683 210
## 15 0.56634953 225
## 16 0.06390017 240
\end{verbatim}

\begin{Shaded}
\begin{Highlighting}[]
\FunctionTok{any}\NormalTok{(wilcox\_nest}\SpecialCharTok{$}\NormalTok{values }\SpecialCharTok{\textless{}=} \FloatTok{0.05}\NormalTok{) }\CommentTok{\#test if any p.values are significantly different: TRUE}
\end{Highlighting}
\end{Shaded}

\begin{verbatim}
## [1] TRUE
\end{verbatim}

\begin{Shaded}
\begin{Highlighting}[]
\NormalTok{dplyr}\SpecialCharTok{::}\FunctionTok{filter}\NormalTok{(wilcox\_nest, values }\SpecialCharTok{\textless{}=} \FloatTok{0.05}\NormalTok{) }\CommentTok{\#print which row has p.value that is significantly different}
\end{Highlighting}
\end{Shaded}

\begin{verbatim}
##       values ind
## 1 0.01047295 165
\end{verbatim}

From the Wilcoxon tests, we see that no median time intervals during the
incubation stage are different between MGSA and HFSA. However, during
the nestling stage, median prey delivery rates at time = 165 are
significantly different between MGSA and HFSA.

\hypertarget{plot-1}{%
\subsection{Plot}\label{plot-1}}

Incubation:

\begin{Shaded}
\begin{Highlighting}[]
\NormalTok{plotInc\_site }\OtherTok{\textless{}{-}} \FunctionTok{ggplot}\NormalTok{(}\AttributeTok{data =}\NormalTok{ ciInc) }\SpecialCharTok{+}
  \FunctionTok{geom\_point}\NormalTok{(}\FunctionTok{aes}\NormalTok{(}\AttributeTok{x =}\NormalTok{ time, }\AttributeTok{y =}\NormalTok{ mean, }\AttributeTok{color =}\NormalTok{ study\_area, }\AttributeTok{group =}\NormalTok{ study\_area),}
             \AttributeTok{position =} \FunctionTok{position\_dodge}\NormalTok{(}\AttributeTok{width=}\FloatTok{0.75}\NormalTok{)) }\SpecialCharTok{+}
  \FunctionTok{geom\_errorbar}\NormalTok{(}\FunctionTok{aes}\NormalTok{(}\AttributeTok{x=}\NormalTok{time, }\AttributeTok{ymax =}\NormalTok{ ci\_h, }\AttributeTok{ymin=}\NormalTok{ci\_l, }\AttributeTok{color =}\NormalTok{ study\_area, }
                    \AttributeTok{group =}\NormalTok{ study\_area),}
                \AttributeTok{position =} \FunctionTok{position\_dodge}\NormalTok{(}\AttributeTok{width=}\FloatTok{0.75}\NormalTok{)) }\SpecialCharTok{+}
  \FunctionTok{labs}\NormalTok{(}\AttributeTok{x =} \StringTok{"Time After Sunset (minutes)"}\NormalTok{, }\AttributeTok{y =} \StringTok{"Mean Prey Deliveries"}\NormalTok{, }
       \AttributeTok{title =} \StringTok{"Incubation"}\NormalTok{, }\AttributeTok{color =} \StringTok{"Study Area"}\NormalTok{) }\SpecialCharTok{+}
  \FunctionTok{theme\_minimal}\NormalTok{() }\SpecialCharTok{+}
  \FunctionTok{theme}\NormalTok{(}\AttributeTok{plot.title =} \FunctionTok{element\_text}\NormalTok{(}\AttributeTok{hjust =} \FloatTok{0.5}\NormalTok{))}
\end{Highlighting}
\end{Shaded}

Nestling (including star indicating significant Wilcoxon test at t =
165):

\begin{Shaded}
\begin{Highlighting}[]
\NormalTok{plotNest\_site }\OtherTok{\textless{}{-}} \FunctionTok{ggplot}\NormalTok{(}\AttributeTok{data =}\NormalTok{ ciNest) }\SpecialCharTok{+}
  \FunctionTok{geom\_point}\NormalTok{(}\FunctionTok{aes}\NormalTok{(}\AttributeTok{x =}\NormalTok{ time, }\AttributeTok{y =}\NormalTok{ mean, }\AttributeTok{color =}\NormalTok{ study\_area, }\AttributeTok{group =}\NormalTok{ study\_area),}
             \AttributeTok{position =} \FunctionTok{position\_dodge}\NormalTok{(}\AttributeTok{width=}\FloatTok{0.75}\NormalTok{)) }\SpecialCharTok{+}
  \FunctionTok{geom\_errorbar}\NormalTok{(}\FunctionTok{aes}\NormalTok{(}\AttributeTok{x=}\NormalTok{time, }\AttributeTok{ymax =}\NormalTok{ ci\_h, }\AttributeTok{ymin=}\NormalTok{ci\_l, }\AttributeTok{color =}\NormalTok{ study\_area, }
                    \AttributeTok{group =}\NormalTok{ study\_area),}
                \AttributeTok{position =} \FunctionTok{position\_dodge}\NormalTok{(}\AttributeTok{width=}\FloatTok{0.75}\NormalTok{)) }\SpecialCharTok{+}
  \FunctionTok{labs}\NormalTok{(}\AttributeTok{x =} \StringTok{"Time After Sunset (minutes)"}\NormalTok{, }\AttributeTok{y =} \StringTok{"Mean Prey Deliveries"}\NormalTok{, }
       \AttributeTok{title =} \StringTok{"Nestling"}\NormalTok{, }\AttributeTok{color =} \StringTok{"Study Area"}\NormalTok{) }\SpecialCharTok{+}
  \FunctionTok{theme\_minimal}\NormalTok{() }\SpecialCharTok{+}
  \FunctionTok{theme}\NormalTok{(}\AttributeTok{plot.title =} \FunctionTok{element\_text}\NormalTok{(}\AttributeTok{hjust =} \FloatTok{0.5}\NormalTok{)) }\SpecialCharTok{+}
  \FunctionTok{geom\_point}\NormalTok{(}\FunctionTok{aes}\NormalTok{(}\AttributeTok{x =} \DecValTok{165}\NormalTok{, }\AttributeTok{y =} \FloatTok{2.0}\NormalTok{), }\AttributeTok{shape =} \DecValTok{8}\NormalTok{, }\AttributeTok{stroke =} \FloatTok{0.1}\NormalTok{)}
\end{Highlighting}
\end{Shaded}

The final plot compares both incubation and nestling stages.

\begin{Shaded}
\begin{Highlighting}[]
\FunctionTok{grid.arrange}\NormalTok{(plotInc\_site, plotNest\_site)}
\end{Highlighting}
\end{Shaded}

\begin{figure}
\centering
\includegraphics{pd_analysis_files/figure-latex/unnamed-chunk-2-1.pdf}
\caption{Mean prey deliveries during the incubation stage (top) and
nestling stage (bottom). Missouri Gulch Study Area (MGSA) is shown in
bue and Hayman Fire Study Area (HFSA) in red. Stars indicate significant
differences in medians based on unpaired two-sample Wilcoxon tests.}
\end{figure}

  \bibliography{references.bib}

\end{document}
